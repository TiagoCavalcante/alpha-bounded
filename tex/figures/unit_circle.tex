\begin{figure}[H]
  \begin{center}
    \begin{tikzpicture}
      \fill (0,0) circle (0.07);
      \node at ([shift={(1cm, -0.5cm)}] 0,0) {$(0, 0, \dots)$};

      % Constants
      \def\n{10} % Number of surrounding points
      \def\k{3.2}  % Constant k in the angle formula
      \def\startOffset{-77}

      % Loop to place points in a circular arrangement
      \foreach \i in {1,...,\n} {
          % Calculate angle for the point using (2 * pi / k) * log(N)
          \pgfmathsetmacro{\angle}{\startOffset + (360 / \k) * ln(\i + 1)}
          \pgfmathsetmacro{\angleout}{\startOffset + (360 / \k) * ln(\i)}


          % Coordinates of the surrounding point at radius 2
          \path (\angle:5) coordinate (P\i);
          \draw[dashed] (0, 0) -- (P\i);
          \ifnum\i>1
            \pgfmathtruncatemacro{\prev}{\i-1}
            % Calculate the start and end angles for each arc
            \pgfmathsetmacro{\startAngle}{\startOffset + (360 / \k) * ln(\prev + 1)}
            \pgfmathsetmacro{\endAngle}{\angle}
            \ifnum\i<4
              \draw[dashed]
              (P\prev) arc[start angle=\startAngle, end angle=\endAngle, radius=5]
              node[midway, above right] {\(\sqrt{2}\)}; % Label at midpoint of each edge
            \else
              \ifnum\i>4
                \draw[dashed]
                (P\prev) arc[start angle=\startAngle, end angle=\endAngle, radius=5];
              \else
                \draw[dashed]
                (P\prev) arc[start angle=\startAngle, end angle=\endAngle, radius=5]
                node[midway, above] {\(\sqrt{2}\)}; % Label at midpoint of each edge
              \fi
            \fi
          \fi

          \ifnum\i<5
            \fill[font=\bfseries] (P\i) circle (0.05) node[above right] {$s_{\i}$};
          \else
            \ifnum\i>9
              \fill (P\i) circle (0.05) node[below] {$s_{\i}$};
            \else
              \fill (P\i) circle (0.05) node[above left] {$s_{\i}$};
            \fi
          \fi
        }

      \node at ([shift={(0.5cm, -0.7cm)}] P10) {$\ddots$};

    \end{tikzpicture}
    \captionsetup{justification=centering}
    \caption{Sequência de pontos de $l^2(\mathbb{Z}_{> 0})$ na esfera unitária. A escala dos angulos que os pontos $s_n$ formam com o eixo $x$ é logarítimica em $n$.}
  \end{center}
\end{figure}
