\phantomsection
\section*{
  \pt{Considerações Finais}
  \en{Final Considerations}
 }
\addcontentsline{toc}{section}{
  \pt{Considerações Finais}
  \en{Final Considerations}
}
\vspace{1cm}

\pt{Como mencionado, pode-se provar uma versão mais forte do limite inferior do teorema \ref*{th:asymptotic} aplicado a uma classe mais ampla de matrizes de Toeplitz que não requer que os coeficientes da matriz assumam valores apenas em $\{0, 1\}$. Isso pode ser combinado com uma análise dos casos de igualdade da desigualdade de Bernstein, que afirma que se $a$ é um polinômio trigonométrico com coeficientes de Fourier $\{a_n\}$, então $\|a\|_\infty \le \sum_{n \in \mathbb{Z}} \lvert a_n\rvert$, para provar uma versão mais forte da caracterização assintótica, novamente aplicada a matrizes de Toeplitz mais gerais.}

\en{As mentioned, a stronger version of the lower bound in Theorem \ref*{th:asymptotic} can be proven when applied to a broader class of Toeplitz matrices that do not require the matrix coefficients to take values only in $\{0, 1\}$. This can be combined with an analysis of the equality cases in Bernstein's inequality, which states that if $a$ is a trigonometric polynomial with Fourier coefficients $\{a_n\}$, then $\|a\|_\infty \le \sum_{n \in \mathbb{Z}} \lvert a_n\rvert$, to prove a stronger asymptotic characterization, again applied to more general Toeplitz matrices.}

\pt{Também é possível estudar as propriedades topológicas das funções $(\alpha, \lambda)$-limitadas, possivelmente entendendo se é o caso que o espaço topológico induzido pelo espaço vetorial das aplicações $(\alpha, \lambda)$-limitadas entre dois espaços métricos herda propriedades dos espaços topológicos induzidos por esses espaços métricos.}
\en{It is also possible to study the topological properties of $(\alpha, \lambda)$-bounded functions, potentially understanding whether the topological space induced by the vector space of $(\alpha, \lambda)$-bounded maps between two metric spaces inherits properties from the topological spaces induced by those metric spaces.}

\pt{É evidente que também é desejável a caracterização assintótica da quantidade de funções $k$-limitadas de $[n]^d$ para $[n]$, com $d$ fixo. Acreditamos que o uso de matrizes de Toeplitz de blocos podem ser úteis para obter esse resultado.}
\en{It is evident that an asymptotic characterization of the number of $k$-bounded functions from $[n]^d$ to $[n]$, with fixed $d$, is also desirable. We believe that the use of block Toeplitz matrices may be useful in obtaining this result.}

\pt{Apesar de não ser utilizada neste trabalho, acreditamos que a Teoria Espectral dos Grafos pode ser usada em conjunto com matrizes de Toeplitz para provar uma ampla gama de resultados acerca da caracterização assintótica de soluções de sistemas de recorrências, uma vez que, ao fornecer informações sobre os autovalores das matrizes associadas à esses sistemas, permite uma abordagem mais direta das aproximações envolvidas. Em particular acreditamos que essa teoria é essencial ao abordar resultados mais gerais sobre por exemplo a quantidade de funções $k$-limitadas de $[n]^n$ para $[n]$ e para $[n]^n$.}
\en{Although not utilized in this work, we believe that Spectral Graph Theory can be used in conjunction with Toeplitz matrices to prove a wide range of results concerning the asymptotic characterization of solutions to recurrence systems. By providing information about the eigenvalues of the matrices associated with these systems, it allows for a more direct approach to the involved approximations. In particular, we believe that this theory is essential when addressing more general results, such as the number of $k$-bounded functions from $[n]^n$ to $[n]$ and to $[n]^n$, for example.}
