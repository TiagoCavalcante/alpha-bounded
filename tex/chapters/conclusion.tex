\phantomsection
\section*{Considerações Finais}
\addcontentsline{toc}{section}{Considerações Finais}
\vspace{1cm}

Como mencionado, pode-se provar uma versão mais forte do limite inferior do teorema \ref*{th:asymptotic} aplicado a uma classe mais ampla de matrizes de Toeplitz que não requer que os coeficientes da matriz assumam valores apenas em $\{0, 1\}$. Isso pode ser combinado com uma análise dos casos de igualdade da desigualdade de Bernstein, que afirma que se $a$ é um polinômio trigonométrico com coeficientes de Fourier $\{a_n\}$, então $\|a\|_\infty \le \sum_{n \in \mathbb{Z}} \lvert a_n\rvert$, para provar uma versão mais forte da caracterização assintótica, novamente aplicada a matrizes de Toeplitz mais gerais.

Também é possível estudar as propriedades topológicas das funções $(\alpha, \lambda)$-limitadas, possivelmente entendendo se é o caso que o espaço topológico induzido pelo espaço vetorial das aplicações $(\alpha, \lambda)$-limitadas entre dois espaços métricos herda propriedades dos espaços topológicos induzidos por esses espaços métricos.

É evidente que também é desejável a caracterização assintótica da quantidade de funções $k$-limitadas de $[n]^d$ para $[n]$, com $d$ fixo. Acreditamos que o uso de matrizes de Toeplitz de blocos podem ser úteis para obter esse resultado.

Apesar de não ser utilizada neste trabalho, acreditamos que a Teoria Espectral dos Grafos pode ser usada em conjunto com matrizes de Toeplitz para provar uma ampla gama de resultados acerca da caracterização assintótica de soluções de sistemas de recorrências, uma vez que, ao fornecer informações sobre os autovalores das matrizes associadas à esses sistemas, permite uma abordagem mais direta das aproximações envolvidas. Em particular acreditamos que essa teoria é essencial ao abordar resultados mais gerais sobre por exemplo a quantidade de funções $k$-limitadas de $[n]^n$ para $[n]$ e para $[n]^n$.
