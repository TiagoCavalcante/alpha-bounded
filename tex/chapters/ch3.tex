\section{Contagem de funções \texorpdfstring{$k$}{k}-limitadas em \texorpdfstring{$[n]$}{[n]}} \vspace{1cm} Muitos problemas em combinatória podem ser descritos como um sistema de recorrências lineares homogêneas de primeira ordem, como mostrado em \cite{coulson}. Todo sistema de recorrências lineares homogêneas de primeira ordem pode ser expresso na forma de matriz. Às vezes, também é útil considerar sequências desses sistemas em $n$ recorrências, bem como entender seus comportamentos assintóticos à medida que $n \to \infty$, mas isso geralmente é uma tarefa difícil. Os resultados deste capítulo são úteis para entender o comportamento assintótico de sistemas cuja matriz correspondente é uma matriz de Toeplitz.

Para cada par de inteiros positivos $n$ e $k$, $a(n,k)$ denota a quantidade de funções $k$-limitadas de $[n]$ para $[n]$.

Nós utilizaremos a notação $\chi_k(i, j) = \begin{cases} 1 & \text{se } \lvert i - j\rvert \le k \\ 0 & \text{se } \lvert i - j\rvert > k \end{cases}$.

Também denotaremos por $M_{n,k} \in \mathbb{Z}^{n \times n}$ a matriz definida por $m_{ij} = \chi_k(i, j)$.

\begin{theorem} \label{th:matrix-form-n-to-n} $a(n,k)$, para $0 \le k \le n - 1$, é dado por $\mathbf{1}_n^\intercal M_{n,k}^{n-1} \mathbf{1}_n$. \end{theorem}

\begin{proof}

  Provaremos o fato mais forte de que o número de funções $k$-limitadas $f : [r] \to [n]$, $r \in [n]$, é dado por $\mathbf{1}_n^\intercal M_{n,k}^{r - 1} \mathbf{1}_n$, e que o número de tais funções com $f(r) = i$ é dado por $(M_{n,k}^{r - 1} \mathbf{1}_n)_i$. Procedemos por indução em $r$.

  De fato, se $r = 1$, toda função $f : [r] \to [n]$ é constante, portanto há $n$ tais funções (uma para cada elemento de $[n]$), e $n = \mathbf{1}_n^\intercal I \mathbf{1}_n = \mathbf{1}_n^\intercal M_{n,k}^{1 - 1} \mathbf{1}_n$, sendo $I$ a matriz identidade $n \times n$, como queríamos.

  Se isso é válido para $r - 1$, então podemos encontrar o número de funções $f : [r] \to [n]$ contando o número de funções $k$-limitadas $g : [r-1] \to [n]$ das quais elas podem ser extensões. Se $f(r) = i$, então $f(r - 1) \in \{i - k, i - k + 1, \dots, i + k\} \cap [n]$, portanto a quantidade de tais funções $f$ é dada pela soma da quantidade de funções $k$-limitadas $g : [r-1] \to [n]$ com $g(r-1) \in \{i - k, i - k + 1, \dots, i + k\} \cap [n]$, que é $\sum_{j = 0}^{n-1} \chi_k(i, j) (M_{n,k}^{r-1} \mathbf{1}_n)_i$, que pela definição de multiplicação de matrizes é igual a $(M_{n,k} (M_{n,k}^{r-1} \mathbf{1}_n))_i = (M_{n,k}^r \mathbf{1}_n)_i$. Consequentemente, o número de todas as funções $k$-limitadas $f : [r] \to [n]$ é dado por $\mathbf{1}_n^\intercal M_{n,k}^r \mathbf{1}_n$, e o resultado segue para $n$, como queríamos.

\end{proof}

Abaixo estão alguns valores usando esta fórmula.

\begin{table}[H]
  \centering
  \adjustbox{width=\textwidth}{
    \begin{tabular}{cccccccccccc}
      \toprule
      \diagbox{$n$}{$k$} & $0$  & $1$       & $2$         & $3$          & $4$           & $5$            & $6$            & $7$             & $8$             & $9$             & $10$            \\
      \midrule
      $2$                & $2$  & $4$       &             &              &               &                &                &                 &                 &                 &                 \\
      $3$                & $3$  & $17$      & $27$        &              &               &                &                &                 &                 &                 &                 \\
      $4$                & $4$  & $68$      & $178$       & $256$        &               &                &                &                 &                 &                 &                 \\
      $5$                & $5$  & $259$     & $1161$      & $2309$       & $3125$        &                &                &                 &                 &                 &                 \\
      $6$                & $6$  & $950$     & $7310$      & $20650$      & $35954$       & $46656$        &                &                 &                 &                 &                 \\
      $7$                & $7$  & $3387$    & $44787$     & $182349$     & $408623$      & $654797$       & $823543$       &                 &                 &                 &                 \\
      $8$                & $8$  & $11814$   & $267802$    & $1569700$    & $4609826$     & $9041498$      & $13667858$     & $16777216$      &                 &                 &                 \\
      $9$                & $9$  & $40503$   & $1569883$   & $13249619$   & $51374875$    & $123984439$    & $222457927$    & $321839625$     & $387420489$     &                 &                 \\
      $10$               & $10$ & $136946$  & $9051480$   & $109906170$  & $561224422$   & $1687725824$   & $3592094120$   & $6040291396$    & $8441614754$    & $100000000000$  &                 \\
      $11$               & $11$ & $5035745$ & $566193771$ & $9871777663$ & $66390984341$ & $249898450197$ & $634363172685$ & $1235414492037$ & $1976196614909$ & $2685194913379$ & $3138428376721$ \\
      % \bottomrule
    \end{tabular}
  }
  \caption{
  \pt{Número de funções $f \colon [n] \to [n]$ $k$-limitadas}
  \en{Number of $k$-bounded functions $f \colon [n] \to [n]$}
  }
  \label{data2}
\end{table}


O código para calcular esses valores pode ser encontrado em \cite{github}. Valores muito maiores podem ser computados. Por exemplo, $a(200,199)$ (que resulta em um número com $463$ dígitos) pode ser computado em menos de $15$ segundos em um computador moderno.

Agora apresentamos a caracterização assintótica desta fórmula.

\begin{theorem} \label{th:asymptotic} Seja $k$ um inteiro positivo fixo. Então, $a(n, k) \sim n(2k+1)^{n-1}$. \end{theorem} \begin{proof} Claramente, $M_{n,k}$ é a matriz de adjacência de um grafo com $n$ vértices, com o grau médio $\overline{d}$ igual a $(2k + 1) - \mathcal{O}(\frac{1}{n})$. Agora, pela desigualdade de Ërdos-Simonovits (veja \cite{erdossimonovits}), $n \overline{d}^{n-1} \le w_{n-1}$, onde $w_{n-1}$ é a quantidade de caminhos com comprimento $n-1$ no grafo. É fácil ver que $w_{n-1} = a(n, k)$. Portanto, $n ((2k + 1) - \mathcal{O}(\frac{1}{n}))^{n-1} \le a(n, k)$.

  Seja \[M_k = \begin{pmatrix} a_{0} & a_{-1} & a_{-2} & \cdots \\ a_{1} & a_{0} & a_{-1} & \cdots \\ a_2 & a_{1} & a_{0} & \cdots \\ \vdots & \vdots & \vdots & \ddots \end{pmatrix}\] a matriz infinita definida por $m_{ij} = \chi_k(i, j)$.

  Claramente apenas uma quantidade finita de termos de $\{a_n\}$ são não nulos, e consequentemente os termos de $\{ a_n \}$ são os coeficientes de Fourier do polinômio trigonométrico $a(x) = \sum_{n = -N}^N a_n e^{i n x}$, e pelo teorema dos operadores limitados induzidos por matrizes de Toeplitz mencionado no capítulo anterior, $M_k$ define um operador limitado em $l^2(\mathbb{Z}_{> 0})$, então $\|M_k\|_2 = \| a \|_{\infty}$ existe, já que todo polinômio trigonométrico é contínuo e consequentemente limitado em $\mathbb{T}$.

  Agora, pela desigualdade de Cauchy e a propriedade submultiplicativa de $\| \cdot \|_2$, temos que
  \begin{align*}
    \mathbf{1}_n^\intercal M_{n,k}^{n-1} \mathbf{1}_n & = \lvert \mathbf{1}_n^\intercal M_{n,k}^{n-1} \mathbf{1}_n\rvert      \\
                                                      & \le \| \mathbf{1}_n^\intercal \|_2 \| M_{n,k}^{n-1} \mathbf{1}_n \|_2 \\
                                                      & = \sqrt{n} \| M_{n,k}^{n-1} \mathbf{1}_n \|_2                         \\
                                                      & \le \sqrt{n} \| M_{n,k}^{n-1} \|_2 \| \mathbf{1}_n \|_2               \\
                                                      & = n \| M_{n,k}^{n-1} \|_2                                             \\
                                                      & \le n \| M_{n,k} \|_2^{n-1}
  \end{align*}

  Pelo resultado provado em, por exemplo, \cite[p. 52]{bottcher}, temos que $\| M_{n,k} \| \le \| M_k \|$.

  Mas os coeficientes de Fourier de $a$ são os coeficientes de Fourier do kernel de Dirichlet $D_k$, e $\| a \|_{\infty} = \esssup{t \in T} \lvert a(t)\rvert = \sup_{t \in T} \lvert a(t)\rvert$. É fato conhecido que $\sup_{t \in T} \lvert D_k(t)\rvert = 2k + 1$. Portanto, $\| M_k \| = 2k + 1$.

  Concluímos a prova aplicando o teorema do confronto a ambas as desigualdades. \end{proof}

Observe que o limite inferior da prova acima pode ser facilmente estendido a qualquer matriz que seja simétrica e tenha entradas não negativas usando a desigualdade de Blakley-Roy (veja \cite{blakley-roy}).

Note que o limite inferior poderia também ser estabelecido usando somente resultados referentes à matrizes de Toeplitz, mas optamos por apresentar a prova aqui presente devido a maior facilidade da sua adaptação a uma gama maior de matrizes.
