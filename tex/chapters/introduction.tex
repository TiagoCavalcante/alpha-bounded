\phantomsection
\section*{
  \pt{Introdução}
  \en{Introduction}
 }
\addcontentsline{toc}{section}{
  \pt{Introdução}
  \en{Introduction}
} % Adds "Introduction" to the ToC without a chapter number
\vspace{1cm}

\normalsize

\pt{É fato conhecido que toda função definida em um subconjunto discreto de $\mathbb{R}$ é contínua considerando a norma induzida. Desta forma, o conceito de continuidade não é muito útil ao analisar funções definidas em conjuntos discretos, motivando assim a introdução de um novo conceito, que chamaremos de funções $(\alpha,\lambda)$-limitadas.}
\en{It is a well-known fact that every function defined on a discrete subset of $\mathbb{R}$ is continuous when considering the induced norm. Thus, the concept of continuity is not very useful when analyzing functions defined on discrete sets, thereby motivating the introduction of a new concept, which we will call $(\alpha,\lambda)$-bounded functions.}

\pt{A introdução de propriedades alternativas para funções em espaços discretos não é inédita na literatura matemática. Conceitos como funções de variação limitada e funções de descontinuidade controlada têm sido explorados para analisar comportamentos específicos em contextos combinatórios e algorítmicos (veja por exemplo \cite{Rosenfeld}). No entanto, as funções $(\alpha,\lambda)$-limitadas se distinguem por sua flexibilidade em parametrizar a restrição de variação, permitindo uma análise mais granular.}
\en{The introduction of alternative properties for functions in discrete spaces is not new in mathematical literature. Concepts such as bounded variation functions and controlled discontinuity functions have been explored to analyze specific behaviors in combinatorial and algorithmic contexts (see, for example, \cite{Rosenfeld}). However, $(\alpha,\lambda)$-bounded functions distinguish themselves by their flexibility in parameterizing variation constraints, allowing for a more granular analysis.}

\pt{Além disso, a utilização de matrizes de Toeplitz na análise de propriedades funcionais tem sido amplamente estudada em áreas como processamento de sinais e teoria dos operadores, como mostrado em \cite{GrenanderSzego}. Trabalhos como \cite{bottcher} demonstram a eficácia dessas matrizes na caracterização de operadores e na obtenção de estimativas assintóticas. Este estudo aproveita essas ferramentas para investigar a contagem assintótica de funções $(k,1)$-limitadas, estabelecendo mais uma ponte entre essas áreas aparentemente distintas.}
\en{Furthermore, the use of Toeplitz matrices in analyzing functional properties has been extensively studied in areas such as signal processing and operator theory, as shown in \cite{GrenanderSzego}. Works like \cite{bottcher} demonstrate the effectiveness of these matrices in characterizing operators and obtaining asymptotic estimates. This study leverages these tools to investigate the asymptotic counting of $(k,1)$-bounded functions, thereby establishing another bridge between these seemingly distinct areas.}

\pt{No âmbito da ciência da computação, onde as funções frequentemente operam em espaços discretos e finitos, como alocação de memória, compressão de dados e processamento de sinais digitais, a introdução das funções $(\alpha,\lambda)$-limitadas fornece outra ferramenta para formalizar e analisar a estabilidade e robustez de algoritmos e sistemas. Elas podem ser particularmente vantajosas em cenários onde variações leves de entrada devido a fatores como ruído, erros de compressão de dados ou pequenas perturbações nos processos computacionais não devem afetar desproporcionalmente as saídas.}
\en{In the realm of computer science, where functions often operate in discrete and finite spaces, such as memory allocation, data compression, and digital signal processing, the introduction of $(\alpha,\lambda)$-bounded functions provides another tool to formalize and analyze the stability and robustness of algorithms and systems. They might be particularly advantageous in scenarios where slight input variations due to factors like noise, data compression errors, or minor perturbations in the computational processes should not disproportionately affect the outputs.}

\pt{A capacidade de contar e caracterizar funções definidas em espaços discretos com restrições específicas, como as impostas pelas funções $(\alpha,\lambda)$-limitadas, é crucial para o desenvolvimento de resultados que exigem controle preciso sobre a variação das funções. Além disso, a análise assintótica da quantidade de funções $(k,1)$-limitadas fornece ideias valiosas sobre o comportamento de sistemas discretos à medida que seu tamanho cresce, permitindo a identificação de padrões e tendências que podem não ser evidentes em análises de casos específicos. Essa compreensão é essencial para o avanço de teorias combinatórias e para a resolução de problemas que envolvem grandes estruturas discretas. O uso de matrizes de Toeplitz e a aplicação da teoria dos grafos neste contexto destacam a combinação de diferentes áreas necessária para abordar problemas complexos em matemática.}
\en{The ability to count and characterize functions defined on discrete spaces with specific constraints, such as those imposed by $(\alpha,\lambda)$-bounded functions, is crucial for developing results that require precise control over function variation. Additionally, the asymptotic analysis of the number of $(k,1)$-bounded functions provides valuable insights into the behavior of discrete systems as their size grows, allowing the identification of patterns and trends that may not be evident in analyses of specific cases. This understanding is essential for advancing combinatorial theories and for solving problems involving large discrete structures. The use of Toeplitz matrices and the application of graph theory in this context highlight the combination of different areas necessary to address complex problems in mathematics.}

\pt{No primeiro capítulo, introduzimos essa nova definição, mostramos alguns exemplos e provamos resultados simples que mostram o contraste entre funções $(\alpha,\lambda)$-limitadas e funções contínuas.}
\en{In the first chapter, we introduce this new definition, present some examples, and prove simple results that highlight the contrast between $(\alpha,\lambda)$-bounded functions and continuous functions.}

\pt{No segundo capítulo, introduzimos a definição de matrizes de Toeplitz, juntamente com resultados relevantes que são empregados no terceiro capítulo, assim como em possíveis generalizações dos teoremas apresentados nele.}
\en{In the second chapter, we introduce the definition of Toeplitz matrices, along with relevant results that are employed in the third chapter, as well as in possible generalizations of the theorems presented therein.}

\pt{No terceiro capítulo, estendemos um resultado de \cite{coulson} ao encontrar a caracterização assintótica da quantidade de funções $(k,1)$-limitadas em $[n]$:}
\en{In the third chapter, we extend a result from \cite{coulson} by finding the asymptotic characterization of the number of $(k,1)$-bounded functions on $[n]$:}

\begin{theorem*}
  \pt{Seja $k \ge 0$ um inteiro fixo, e seja $a(n, k)$ a quantidade de funções $k$-limitadas de $[n]\to[n]$. Então, $a(n,k) \sim n(2k+1)^{n-1}$.}
  \en{Let $k \ge 0$ be a fixed integer, and let $a(n, k)$ be the number of $k$-bounded functions from $[n] \to [n]$. Then, $a(n,k) \sim n(2k+1)^{n-1}$.}
\end{theorem*}

\pt{Este resultado é provado usando propriedades de matrizes de Toeplitz e uma cota bem conhecida da teoria dos grafos.}
\en{This result is proven using properties of Toeplitz matrices and a well-known bound from graph theory.}

\pt{Por fim, no apêndice apresentamos alguns resultados interessantes acerca de matrizes de Toeplitz.}
\en{Finally, in the appendix, we present some interesting results concerning Toeplitz matrices.}

\pt{Desejamos com este trabalho mostrar como uma combinação de matrizes de Toeplitz e teoria dos grafos podem ser usadas para obter estimativas assintóticas precisas de quantidades como a estudada neste trabalho: o número funções $k$-limitadas de $[n]$ para $[n]$.}
\en{With this work, we aim to demonstrate how a combination of Toeplitz matrices and graph theory can be used to obtain precise asymptotic estimates of quantities such as those studied in this work: the number of $k$-bounded functions from $[n]$ to $[n]$.}
