\section{Resultados da teoria de matrizes de Toeplitz} \vspace{1cm}

\par Em \cite{Toeplitz1911} Otto Toeplitz apresenta as $L$-formas, anteriormente estudadas em sua tese de habilitação para a universidade Universidade de Göttingen. Posteriormente seus estudos foram adaptados para o que hoje chamamos de matrizes de Toeplitz.

Uma matriz de Toeplitz é uma matriz com entradas complexas possivelmente infinita da forma \[ \begin{pmatrix} a_{0} & a_{-1} & a_{-2} & \cdots \\ a_{1} & a_{0} & a_{-1} & \cdots \\ a_2 & a_{1} & a_{0} & \cdots \\ \vdots & \vdots & \vdots & \ddots \end{pmatrix}\text{.} \]

As matrizes de Toeplitz possuem uma ampla gama de aplicações, que vão desde teoria dos operadores, análise harmonica, teoria da informação e processamento de sinais à mecânica quântica.

Essas matrizes são especialmente úteis quando a sequência $\{a_n\}$ determina os coeficientes de Fourier de uma determinada função, uma vez que isso possibilita utilizar ferramentas da análise para calcular, por exemplo, a norma do operador em $l^2(\mathbb{Z}_{> 0})$ que a matriz representa.

Em \cite{Toeplitz1911} Toeplitz mostra uma condição necessária e suficiente para que uma matriz de Toeplitz induza um operador limitado em $l^2(\mathbb{Z}_{> 0})$, é claro, utilizando a linguagem das $L$-formas.

\begin{theorem*}[Operadores limitados induzidos por matrizes de Toeplitz]
  A matriz infinita \[A = \begin{pmatrix} a_{0} & a_{-1} & a_{-2} & \cdots \\ a_{1} & a_{0} & a_{-1} & \cdots \\ a_2 & a_{1} & a_{0} & \cdots \\ \vdots & \vdots & \vdots & \ddots \end{pmatrix}\] induz um operador limitado em $l^2(\mathbb{Z}_{>0})$ se e somente se os números $\{a_n\}$ são os coeficientes de Fourier de uma função integrável $a \in L^\infty(\mathbb{T})$ definida no círculo unitário complexo, \[a_n = \frac{1}{2\pi} \int_{0}^{2\pi} a(e^{i \theta})e^{-i n \theta} \d{\theta}, \quad n \in \mathbb{Z}\text{.}\]
  Nesse caso, a norma do operador $A$ é igual à \[\|a\|_{\infty} \coloneqq \esssup_{t \in \mathbb{T}}\lvert a(t)\rvert\text{.}\]
\end{theorem*}

Essa formulação e uma prova com notação moderna são encontrados em \cite[p. 1]{bottcher}.

A norma $\| a \|_{\infty}$ mencionada no teorema acima corresponde à norma $L^\infty$, definida como
\[ \esssup_{t \in \mathbb{T}}\lvert a(t)\rvert \coloneqq \inf \{ C \geq 0 \mid |a(t)| \leq C \text{ para quase todo } t \in \mathbb{T} \}\text{.}
\]

Aqui, ``quase todo'' significa que a condição $|a(t)| \leq C$ deve ser satisfeita para todos os $t \in \mathbb{T}$, exceto possivelmente em um conjunto de medida zero. Intuitivamente, a norma $L^\infty$ captura o valor máximo da função $a$ exceto em um conjunto ``irrelevante'' de pontos.

O \textit{símbolo} de uma matriz de Toeplitz é a função $a \in L^\infty(\mathbb{T})$ que gera a matriz através de seus coeficientes de Fourier. Esse símbolo encapsula toda a informação necessária para caracterizar as propriedades espectrais e operacionais da matriz de Toeplitz.

Em diversos casos, pode-se utilizar a norma do operador em $l^2(\mathbb{Z}_{> 0})$ induzido por uma matriz de Toeplitz para estudar o limite da sequência das normas de matrizes que são ou estão relacionadas à matrizes de Toeplitz, ou vice-versa. Grande parte dos teoremas acerca dessa relação são provados usando a teoria das álgebras $C^\ast$, assim como outros resultados da teoria das matrizes de Toeplitz. Essa estratégia é apresentada com profundidade em \cite{bottcher}. Na prova de \ref{th:asymptotic} usamos um resultado básico sobre essa relação.

Vamos agora a um tipo específico de matriz de Toeplitz. As matrizes de banda surgem naturalmente em diversas aplicações, como na discretização de operadores diferenciais e em algoritmos eficientes de processamento de sinais. No contexto das matrizes de Toeplitz, a restrição de banda permite simplificar a análise espectral e facilita a obtenção de estimativas de norma, tornando-as ferramentas valiosas para a análise assintótica e a resolução de sistemas lineares estruturados.

\begin{definition*}
  Uma matriz de Toeplitz $ M $ com entradas complexas é dita \textit{de banda $ k $} se $ m_{ij} = 0 $ para todos $ i \in I$ e $j \in J $, sendo $I$ o conjunto dos índices das linhas de $M$ e $J$ o conjunto dos índices das colunas, tais que $ |i - j| > k $.
\end{definition*}

\begin{theorem}[Norma de matrizes de Toeplitz de banda] \label{norm} Seja $M$ uma matriz de Toeplitz possivelmente infinita com entradas reias e banda $k$. Então, a norma espectral de $M$, denotada por $\| M \|_2$, satisfaz \[ \| M \|_2 \leq (2k + 1) \cdot \max_{|i - j| \leq k} |m_{ij}|. \] \end{theorem}

Para provar o teorema anterior, será necessária a utilização da desigualdade $\|M\|_2 \leq \|M\|_\infty$. A seguir, apresentamos uma prova dessa desigualdade para completude.

As normas de matrizes são normas de operadores, permitindo medir a distorção que uma matriz pode causar em um espaço vetorial. Consideremos um espaço vetorial $V$ equipado com uma norma $\| \cdot \|_p$, conhecida como norma $p$-norma, definida para $1 \leq p \leq \infty$ por
\[
  \| v \|_p = \left( \sum_{i=1}^{n} |v_i|^p \right)^{1/p} \quad \text{para } p < \infty, \quad \text{e} \quad \| v \|_\infty = \max_{1 \leq i \leq n} |v_i|.
\]

As $p$-normas satisfazem várias desigualdades importantes, como a desigualdade de Hölder e a desigualdade de Minkowski, que estabelecem relações entre diferentes normas $p$ e $q$ em espaços vetoriais.

A norma de operador de uma matriz $M \in \mathbb{C}^{n \times n}$ associada à norma $p$-norma no espaço vetorial $V$ é definida por
\[
  \| M \|_{p} = \sup_{\| v \|_p = 1} \| M v \|_p.
\]

Especificamente, para matrizes $ M \in \mathbb{C}^{n \times n} $, a norma espectral $\| M \|_2$ corresponde ao valor singular de $ M $ de maior módulo, enquanto a norma infinito $\| M \|_\infty$ é definida como o maior somatório das magnitudes das entradas em qualquer linha da matriz
\[
  \| M \|_\infty = \max_{1 \leq i \leq n} \sum_{j=1}^{n} |m_{ij}|.
\]

\begin{theorem}
  Se $M \in \mathbb{C}^{n \times n} $ então
  \[
    \| M \|_2 \leq \| M \|_\infty.
  \]
\end{theorem}

\begin{proof}
  Seja $\lambda$ o autovalor de $M$ de maior módulo. Pelo teorema do círculo de Gershgorin, provado pela primeira vez em \cite{Gerschgorin}, temos
  \[
    \lvert \lambda \rvert \in \bigcup_{i = 1}^n \overline{D}\left(m_{ii}, \sum_{j \ne i} \lvert m_{ij} \rvert\right) \subseteq \overline{D}\left(0, \max_{1 \leq i \leq n} \sum_{j=1}^{n} \lvert m_{ij}\rvert\right)
  \] onde $\overline{D}(c, r)$ é o disco complexo fechado de centro $c \in \mathbb{C}$ e raio $r \in \mathbb{R}_{\ge 0}$.

  Portanto, temos que
  \[
    \| M \|_2 = \lvert \lambda \lvert \leq \| M \|_\infty.
  \]
\end{proof}

\begin{proof}[Prova do teorema \ref{norm}]
  Em cada linha $i$, existem no máximo $k$ entradas não nulas acima da diagonal e $k$ abaixo, além da própria entrada na diagonal, totalizando no máximo $2k + 1$ entradas não nulas, correspondentes aos índices $j$ tais que $\lvert i - j \rvert \leq k$.

  Portanto, sendo $I$ o conjunto dos índices das linhas de $M$ e $J$ o conjunto dos índices das colunas, temos \[ \| M \|_2 \le \| M \|_\infty = \sup_{i \in I} \sum_{j \in J} |m_{ij}| \leq (2k + 1) \cdot \max_{|i - j| \leq k} |m_{ij}| \]
\end{proof}

A matriz $ M_{n,k} $ definida no teorema \ref{th:matrix-form-n-to-n} é um exemplo de matriz de Toeplitz de banda $ k $, onde os elementos $ m_{ij} $ são iguais a $ 1 $ se $ |i - j| \le k $ e $ 0 $ caso contrário. A norma espectral dessa matriz satisfaz $ \| M_{n,k} \|_2 = 2k + 1 $, conforme utilizado no teorema \ref{th:asymptotic}.

Desse modo esse teorema pode ser usado para generalizar o limite superior de \ref{th:asymptotic} em casos onde não conseguimos calcular diretamente a norma da matrix de Toeplitz relacionada ao sistema.

Neste capítulo, foram estabelecidos os principais resultados da teoria de matrizes de Toeplitz que fundamentam as análises subsequentes deste trabalho. As propriedades e teoremas apresentados proporcionam as ferramentas matemáticas necessárias para a contagem de funções $k$-limitadas discutida no próximo capítulo.
