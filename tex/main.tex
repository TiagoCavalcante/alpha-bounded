\documentclass[a4paper,12pt]{article}
\usepackage[brazil]{babel}
\usepackage{amsthm, amsmath, amssymb, amsfonts, mathrsfs, mathtools}
\usepackage{xfrac, cancel}
\usepackage{seqsplit}
\usepackage{graphicx, setspace, xcolor, subcaption, svg}
\usepackage{float}
\usepackage[framemethod=TikZ]{mdframed}
\usepackage[bottom]{footmisc}
\usepackage[square,numbers]{natbib}
\usepackage{hyperref}
% \hypersetup{colorlinks=true, linkcolor=blue}
\usepackage{booktabs, diagbox, adjustbox, siunitx}
\usepackage{tabularx}
\usepackage{enumitem}
\usepackage{calc}
\usepackage[left=3cm,top=3cm,right=2cm,bottom=2cm]{geometry}
\pagestyle{myheadings}
\allowdisplaybreaks
\usepackage{setspace}
\usepackage{titlesec}
\usepackage{tocloft} % Package for customizing the table of contents

\addto\captionsbrazil{
  \renewcommand{\proofname}{Prova} % Muda "demonstração" para "prova"
}

% Commands
\renewcommand{\d}[1]{\ensuremath{\operatorname{d}\!{#1}}}
\DeclareMathOperator*{\esssup}{ess\,sup}
\DeclareMathOperator{\tr}{tr}
\renewcommand\qedsymbol{$\blacksquare$}
\newcommand{\textover}[2]{\overset{\mathclap{\scriptstyle #2}}{#1}}

\newtheorem*{theorem*}{Teorema}
\newtheorem{theorem}{Teorema}
\newtheorem{corollary}{Corolário}
\newtheorem{conjecture}{Conjectura}
\newtheorem{example}{Exemplo}
\theoremstyle{definition}
\newtheorem*{definition*}{Definição}
\newtheorem{definition}{Definição}

\renewcommand{\cftsecleader}{\cftdotfill{\cftdotsep}} % Adds dots for sections
\renewcommand{\cftsubsecleader}{\cftdotfill{\cftdotsep}} % Adds dots for subsections
\renewcommand{\cftsubsubsecleader}{\cftdotfill{\cftdotsep}} % Adds dots for subsubsections





\bibliographystyle{abbrvnat}

\begin{document}

\pagenumbering{gobble}
\begin{center}
  \pt{\textbf{INSTITUTO ALPHA LUMEN}\\}
  \en{\textbf{ALPHA LUMEN INSTITUTE}\\}
  \vspace{0.1cm}
  \pt{2° ANO DO ENSINO MÉDIO}
  \en{2\textsuperscript{nd} Year of High School}
\end{center}
\vspace{3cm}
\begin{center}
  Tiago Trindade\\
  Pedro Martineli
\end{center}
\vspace{8cm}
\begin{center}
  \large \textbf{
    \pt{Contagem de funções \texorpdfstring{$k$}{k}-limitadas em \texorpdfstring{$[n]$}{[n]}}
    \en{Counting \texorpdfstring{$k$}{k}-bounded functions on \texorpdfstring{$[n]$}{[n]}}
  }
\end{center}
\vspace{8cm}
\begin{center}
  São José dos Campos - SP\\2024
\end{center}

\begin{center}
  Tiago Trindade\\Pedro Martineli
\end{center}
\vspace{7cm}
\begin{center}
  \large \textbf{Contagem de funções $k$-limitadas em $[n]$}
\end{center}
\vspace{5cm}
\begin{flushright}
  \parbox{0.45\linewidth}{\sloppy
    Trabalho de Conclusão de Curso apresentado ao Instituto Alpha Lumen como parte das                 exigências para a obtenção do título de conclusão da 2ª série do Ensino Médio. \\[10pt]
    Orientador: Prof. Lucas Henrique \\[5pt]
    Supervisora: Prof. Ana Cecília Leal
  }
\end{flushright}
\vspace{4cm}
\begin{center}
  \normalsize São José dos Campos\\2024
\end{center}
\newpage

\vspace*{\fill}

\hspace{8cm}
\begin{minipage}{0.45\textwidth}
  \raggedright
  \pt{Dedicamos este trabalho ao professor Krerley Oliveira, cuja dedicação à educação é evidente em iniciativas como o Novo Ensino Suplementar. Seu apoio e orientações precisas foram cruciais, inspirando-nos continuamente à excelência.}
  \en{We dedicate this work to Professor Krerley Oliveira, whose dedication to education is evident in initiatives such as Novo Ensino Suplementar. His support and precise guidance were crucial, continuously inspiring us towards excellence.}
\end{minipage}

\begin{center}
  \large \textbf{
    \pt{Agradecimentos}
    \en{Acknowledgements}
  }
\end{center}

\par
\pt{Agradecemos profundamente ao professor Lucas Henrique, orientador deste trabalho, cuja expertise e direcionamento foram fundamentais para a elucidação e prova dos resultados aqui apresentados. Sua capacidade de guiar nossos esforços em meio aos desafios teóricos foi essencial para o sucesso deste estudo.}
\en{We deeply thank Professor Lucas Henrique, the advisor of this work, whose expertise and guidance were fundamental for the elucidation and proof of the results presented here. His ability to steer our efforts amidst theoretical challenges was essential for the success of this study.}

\pt{Expressamos nossa sincera gratidão aos nossos pais, que nos proporcionaram suporte constante e condições favoráveis para que seguíssemos com determinação em busca de nossos objetivos. Seu apoio incondicional e sua confiança em nossa capacidade foram pilares em nossa formação pessoal e acadêmica.}
\en{We express our sincere gratitude to our parents, who provided us with constant support and favorable conditions to pursue our goals with determination. Their unconditional support and trust in our abilities were pillars in our personal and academic development.}

\pt{Somos também gratos ao Novo Ensino Suplementar, que nos ofereceu uma base teórica robusta e indispensável, enriquecendo nosso conhecimento e ampliando nossa perspectiva acadêmica, o que foi crucial para o desenvolvimento deste trabalho.}
\en{We are also grateful to Novo Ensino Suplementar, which provided us with a robust and indispensable theoretical foundation, enriching our knowledge and broadening our academic perspective, which was crucial for the development of this work.}

\pt{Agradecemos ao Instituto Alpha Lumen por possibilitar um avanço significativo em nossa trajetória acadêmica. O suporte oferecido pelo instituto foi determinante, permitindo que nos dedicássemos ao aprofundamento nos estudos matemáticos, o que foi decisivo para nosso crescimento e realização acadêmica.}
\en{We thank Instituto Alpha Lumen for enabling significant advancement in our academic journey. The support offered by the institute was decisive, allowing us to dedicate ourselves to the deepening of mathematical studies, which was crucial for our academic growth and achievement.}

\begin{center}
  \large \textbf{Resumo}
\end{center}
\vspace{1cm}
\par Neste trabalho, estabelecemos a definição de funções $(\alpha, \lambda)$-limitadas, encontramos alguns resultados imediatos desta definição e descobrimos a caracterização assintótica da quantidade de funções $k$-limitadas em $[n]$ considerando a distância de Manhattan, utilizando ferramentas da combinatória algébrica, como matrizes de Toeplitz, e da teoria dos grafos.

\vspace{1cm}

\begin{raggedleft}
  \textbf{Palavras-chave}: Espaços Métricos; Matrizes de Toeplitz; Estimativas Assintóticas; Combinatória Algébrica; Teoria dos Grafos.
\end{raggedleft}
\newpage
s
\begin{center}
  \large \textbf{Abstract}
\end{center}
\vspace{1cm}
\par In this work, we introduce the concept of $(\alpha, \lambda)$-bounded functions, characterized by limited local variation, which is especially useful in discrete sets. Initially, we formally define these functions and investigate their fundamental properties, highlighting significant differences from continuous functions. The main result obtained is the asymptotic characterization of $a(n, k)$, representing the number of functions from $[n]$ to $[n]$ that are $k$-bounded with respect to the Manhattan distance. The proof of this result combines Toeplitz matrices with a well-known inequality from graph theory. Additionally, we emphasize the relevance of the intersection among algebraic combinatorics, graph theory, and the analysis of discrete metric spaces, suggesting future research into the topological properties of these functions and their potential applications in computational and mathematical contexts.

\vspace{1cm}

\begin{raggedleft}
  \textbf{Keywords}: metric spaces; Toeplitz matrices; asymptotic estimates; graph theory; algebraic combinatorics.
\end{raggedleft}

\tableofcontents

\newpage

\section*{Notações}
% \addcontentsline{toc}{section}{Notações}

\begin{description}[leftmargin=!, labelwidth=\widthof{\texttt{Função $f : X \to Y$}}]
  \pt{\item[{$[n]$}] Conjunto $\{1, 2, \dots, n\}$.}
  \en{\item[{$[n]$}] Set $\{1, 2, \dots, n\}$.}

  \pt{\item[$\mathbb{Z}_{>0}$] Conjunto dos inteiros positivos.}
  \en{\item[$\mathbb{Z}_{>0}$] Set of positive integers.}

  \pt{\item[$(X, d_X)$] Espaço métrico $X$ com a métrica $d_X$.}
  \en{\item[$(X, d_X)$] Metric space $X$ with metric $d_X$.}

  \pt{\item[$\mathbf{1}_n$] Vetor coluna de dimensão $n$ com todos os elementos iguais a $1$.}
  \en{\item[$\mathbf{1}_n$] Column vector of dimension $n$ with all elements equal to $1$.}

  \pt{\item[$(v)_i$] O $i$-ésimo elemento de um vetor $v$.}
  \en{\item[$(v)_i$] The $i$-th element of a vector $v$.}

  \pt{\item[$\| \cdot \|_2$] Norma espectral de uma matriz ou vetor.}
  \en{\item[$\| \cdot \|_2$] Spectral norm of a matrix or vector.}

  \pt{\item[$H^2(\mathbb{T})$] Espaço de Hardy das funções analíticas na circunferência unitária $\mathbb{T}$.}
  \en{\item[$H^2(\mathbb{T})$] Hardy space of analytic functions on the unit circle $\mathbb{T}$.}

  \pt{\item[$T(a)$] Operador de Toeplitz com símbolo $a \in L^\infty(\mathbb{T})$, definido por $T(a)f = P(af)$ onde $P$ é a projeção de Szegő.}
  \en{\item[$T(a)$] Toeplitz operator with symbol $a \in L^\infty(\mathbb{T})$, defined by $T(a)f = P(af)$ where $P$ is the Szegő projection.}

  \pt{\item[$\sigma(T(a))$] Espectro do operador de Toeplitz $T(a)$.}
  \en{\item[$\sigma(T(a))$] Spectrum of the Toeplitz operator $T(a)$.}

  \pt{\item[$\{a_n\}$] Sequência indexada por $n \in \mathbb{Z}$, ou seja, uma sequência de termos $\{a_n : n \in \mathbb{Z}\}$ onde $n$ varia em todos os inteiros.}
  \en{\item[$\{a_n\}$] Sequence indexed by $n \in \mathbb{Z}$, that is, a sequence of terms $\{a_n : n \in \mathbb{Z}\}$ where $n$ ranges over all integers.}

  \pt{\item[$\Lambda$] Matriz diagonal resultante da diagonalização de uma matriz circulante através da matriz de Fourier.}
  \en{\item[$\Lambda$] Diagonal matrix resulting from the diagonalization of a circulant matrix via the Fourier matrix.}

  \pt{\item[$D_k(t)$] Kernel de Dirichlet de ordem $k$.}
  \en{\item[$D_k(t)$] Dirichlet kernel of order $k$.}

  \pt{\item[$T_n(a)$] Matriz de Toeplitz $n \times n$ gerada pelo símbolo $a \in L^\infty(\mathbb{T})$.}
  \en{\item[$T_n(a)$] $n \times n$ Toeplitz matrix generated by symbol $a \in L^\infty(\mathbb{T})$.}

  \pt{\item[$l^2(\mathbb{Z}_{> 0})$] Espaço de sequências quadrado-somáveis indexadas por $\mathbb{Z}_{> 0}$.}
  \en{\item[$l^2(\mathbb{Z}_{> 0})$] Space of square-summable sequences indexed by $\mathbb{Z}_{> 0}$.}

  \pt{\item[$a(x)$] Função símbolo associada a uma matriz de Toeplitz, geralmente definida como um polinômio trigonométrico.}
  \en{\item[$a(x)$] Symbol function associated with a Toeplitz matrix, usually defined as a trigonometric polynomial.}

  \pt{\item[$L^\infty(\mathbb{T})$] Espaço das funções definidas na circunferência unitária complexa $\mathbb{T}$ essencialmente limitadas.}
  \en{\item[$L^\infty(\mathbb{T})$] Space of essentially bounded functions defined on the complex unit circle $\mathbb{T}$.}

  \pt{\item[$\esssup$] Supremo essencial.}
  \en{\item[$\esssup$] Essential supremum.}

  \pt{\item[$\overline{B}_{\lambda}(x)$] Bola fechada de raio $\lambda$ centrada em $x$ no espaço métrico considerado.}
  \en{\item[$\overline{B}_{\lambda}(x)$] Closed ball of radius $\lambda$ centered at $x$ in the considered metric space.}

  \pt{\item[$a(n) \sim b(n)$] $a(n)$ e $b(n)$ são assintoticamente equivalentes, isto é, $\lim_{n \to \infty} \frac{a(n)}{b(n)} = 1$.}
  \en{\item[$a(n) \sim b(n)$] $a(n)$ and $b(n)$ are asymptotically equivalent, that is, $\lim_{n \to \infty} \frac{a(n)}{b(n)} = 1$.}

  \pt{\item[$f \in L^1(\mathbb{T})$] Função $f$ integrável definida na circunferência unitária $\mathbb{T}$.}
  \en{\item[$f \in L^1(\mathbb{T})$] Integrable function $f$ defined on the complex unit circle $\mathbb{T}$.}

  \pt{\item[$\mathcal{O}(\cdot)$] Notação Big-O, indicando que uma função $f(n)$ cresce no máximo na ordem de outra função $g(n)$, ou seja, $f(n) = \mathcal{O}(g(n))$ se existem constantes $C, n_0 > 0$ tais que $|f(n)| \leq C \cdot |g(n)|$ para todo $n \geq n_0$.}
  \en{\item[$\mathcal{O}(\cdot)$] Big-O notation, indicating that a function $f(n)$ grows at most on the order of another function $g(n)$, that is, $f(n) = \mathcal{O}(g(n))$ if there exist constants $C, n_0 > 0$ such that $|f(n)| \leq C \cdot |g(n)|$ for all $n \geq n_0$.}
\end{description}


\pagenumbering{arabic}
\setcounter{page}{8}
\section*{Introdução}
\addcontentsline{toc}{section}{Introdução} % Adds "Introduction" to the ToC without a chapter number

\vspace{1cm}
\normalsize
É fato conhecido que toda função definida em um subconjunto discreto de $\mathbb{R}$ é contínua considerando a norma induzida. Desta forma, o conceito de continuidade não é muito útil ao analisar funções definidas em conjuntos discretos, motivando assim a introdução de um novo conceito, que chamaremos de funções $(\alpha,\lambda)$-limitadas. No âmbito da ciência da computação, onde as funções frequentemente operam em espaços discretos e finitos, como alocação de memória, compressão de dados e processamento de sinais digitais, a introdução das funções $(\alpha,\lambda)$-limitadas fornece outra ferramenta para formalizar e analisar a estabilidade e robustez de algoritmos e sistemas. Elas podem ser particularmente vantajosas em cenários onde variações leves de entrada devido a fatores como ruído, erros de compressão de dados ou pequenas perturbações nos processos computacionais não devem afetar desproporcionalmente as saídas.

No primeiro capítulo, introduzimos essa nova definição, mostramos alguns exemplos e provamos resultados simples que mostram o contraste entre funções $(\alpha,\lambda)$-limitadas e funções contínuas.

No segundo capítulo abordamos a teoria de matrizes de Toeplitz, apresentado alguns resultados relevantes.

Por fim, no terceiro capítulo, estendemos um resultado de \cite{coulson} ao encontrar a caracterização assintótica da quantidade de funções $(k,1)$-limitadas em $[n]$:

\begin{theorem*} Seja $k \ge 0$ um inteiro fixo, e seja $a(n, k)$ a quantidade de funções $k$-limitadas de $[n]\to[n]$. Então, $a(n,k) \sim n(2k+1)^{n-1}$. \end{theorem*}

Este resultado é provado usando propriedades de matrizes de Toeplitz e uma cota bem conhecida da teoria dos grafos.

Desejamos com este trabalho mostrar como uma combinação de matrizes de Toeplitz e Teoria dos Grafos podem ser usadas para obter estimativas assintóticas precisas de certas quantidades.

\newpage
\section{Definições}
\vspace{1cm}
A definição para funções $(\alpha,\lambda)$-limitadas é inicialmente apresentada como uma propriedade pontual, assim como a definição de continuidade: \begin{definition} Sejam $\alpha, \lambda \ge 0$, $(X, d_X)$ e $(Y, d_Y)$ espaços métricos, $\mathcal{X} \subseteq X$. Uma função $f : \mathcal{X} \to Y$ é dita \textit{$(\alpha,\lambda)$-limitada no ponto $x \in \mathcal{X}$} quando, para todo $\tilde{x} \in \mathcal{X}$ tal que $d_X(x, \tilde{x}) \le \lambda$, temos $d_Y(f(x), f(\tilde{x})) \le \alpha$. \end{definition}

\begin{definition} Sejam $\alpha, \lambda \ge 0$, $(X, d_X)$ e $(Y, d_Y)$ espaços métricos, $\mathcal{X} \subseteq X$. Uma função $f : \mathcal{X} \to Y$ é dita \textit{$(\alpha,\lambda)$-limitada} quando, para todo $x$ em $\mathcal{X}$, $f$ é $(\alpha,\lambda)$-limitada no ponto $x$. \end{definition}

Note que, para $\lambda = 1$, a definição acima estende naturalmente a condição de Hölder para a ordem $0$ e coeficiente $\alpha$ (veja \cite{holder}). Também é importante notar que muitos resultados de funções contínuas não se estendem para funções $(\alpha, \lambda)$-limitadas. Veja o exemplo abaixo.

Ao alterar a métrica, podemos considerar $\lambda = 1$ na definição acima sem perda de generalidade, porque $(X, d_X)$ é um espaço métrico se, e somente se, $(X, \lambda d_X)$ é um espaço métrico para todo $\lambda > 0$. Se desejarmos ter $d_X(x, \tilde{x}) \le \lambda$ para algum $\lambda > 0$, então podemos considerar o espaço métrico $(X, \frac{1}{\lambda} d_X)$ ao invés de $(X, d_X)$. Como nosso principal interesse neste artigo são as funções $(k,1)$-limitadas, usaremos $\alpha$-limitada para denotar $(\alpha,1)$-limitada.

\begin{example} Considere $V$ um conjunto de Vitali. A função $f : [0, 1] \to \mathbb{R}$, definida por $f(x) = 1_V(x)$ se $x \in (0, 1)$ e $f(x) = 1 - x$ caso contrário, é $1$-limitada (considerando a métrica usual) mas não é contínua, não possui pontos fixos e não é Lebesgue integrável (veja \cite{vitali}). \end{example}

Toda função $f : X \to \mathbb{R}$ definida em um subconjunto discreto de $\mathbb{R}$ é contínua considerando a métrica induzida por $\mathbb{R}$, mas pode não ser $\alpha$-limitada para qualquer $\alpha > 0$. Veja a seguir.

\begin{example} \label{two_to_n} $f : \mathbb{Z} \to \mathbb{Z}$ dada por $n \mapsto 2^n$ não é $\alpha$-limitada (considerando a métrica usual) para nenhum $\alpha > 0$, pois $f(n + 1) - f(n) = 2^{n}$ é ilimitada. \end{example}

%% Talk about breaking point

Abaixo estão alguns resultados imediatos da definição de $\alpha$-limitado.

\begin{theorem} Toda função $\alpha$-limitada é $(\alpha+\epsilon)$-limitada para todo $\epsilon > 0$. \end{theorem} \begin{proof} Evidente. \end{proof}

\begin{theorem} Sejam $(X, d_X)$ e $(Y, d_Y)$ espaços métricos, $\mathcal{X} \subseteq X$. Toda função Lipschitz contínua, com constante $K$, $f : \mathcal{X} \to Y$ é $K$-limitada. \end{theorem} \begin{proof} Para todo $x_1 \in \mathcal{X}$, $x_2 \in \mathcal{X}, d_X(x_1, x_2) \le 1$ e consequentemente $d_Y(f(x_1), f(x_2)) \le K d_X(x_1, x_2) \le K$. \end{proof}
Note que nem todas as funções $\alpha$-Hölder contínuas são $\beta$-limitadas para algum $\beta > 0$. A função $\sqrt{\cdot} : \mathbb{R}^+ \to \mathbb{R}$ constitui um contraexemplo (considerando a métrica usual) pelo mesmo motivo mencionado no exemplo \ref{two_to_n}.

\begin{theorem} Sejam $(X, d_X)$ e $(Y, d_Y)$ espaços métricos, $\mathcal{X} \subseteq X$, e $f : \mathcal{X} \to Y$ uma função. Se para todo $\epsilon > 0$ existe um $\alpha \in (0, \epsilon]$ tal que $f$ é $\alpha$-limitada, então $f$ é contínua. \end{theorem} \begin{proof} Com efeito, para cada $x \in \mathcal{X}, \epsilon > 0$, existe algum $\alpha \in (0, \epsilon]$ tal que $\tilde{x} \in \mathcal{X}, d_X(x, \tilde{x}) \le 1 \implies d_Y(f(x), f(\tilde{x})) \le \alpha \le \epsilon$. \end{proof}

\begin{theorem} Sejam $(X, d_X)$ e $(Y, d_Y)$ espaços métricos, $\mathcal{X} \subseteq X$. Se $(f_n) \to f$ ponto a ponto, $f_n : \mathcal{X} \to Y$ são $\alpha$-limitadas para todo $n \in \mathbb{N}$, então $f$ é $\alpha$-limitada. \end{theorem} \begin{proof} Suponha, por contradição, que $f$ não seja $\alpha$-limitada. Então existem $x_1, x_2 \in \mathcal{X}$, $d_X(x_1, x_2) \le 1$, tal que $d_Y(f(x_1), f(x_2)) \coloneqq k > \alpha$. Como $(f_n) \to f$, existe algum $n \in \mathbb{N}$ com $d_Y(f_n(x_1), f(x_1)) < \frac{k - \alpha}{2}$ e $d_Y(f_n(x_2), f(x_2)) < \frac{k - \alpha}{2}$. Consequentemente, $d_Y(f_n(x_1), f_n(x_2)) > k - \frac{k - \alpha}{2} - \frac{k - \alpha}{2} = \alpha$, uma contradição. \end{proof}

As funções $\alpha$-limitadas podem não ser o limite pontual de qualquer sequência de funções contínuas.

\begin{example} A função de Dirichlet $1_\mathbb{Q} : [0, 1] \to \mathbb{R}$ é $1$-limitada, considerando a métrica usual de $\mathbb{R} \supset [0, 1]$, mas não é o limite pontual de nenhuma sequência de funções contínuas, pois é uma função da classe 2 de Baire (veja \cite{baire}). \end{example}

A aritmética com funções $\alpha$-limitadas não é a mesma que com funções contínuas definidas em $\mathbb{R}$.

\begin{theorem} Sejam $(X, d_X)$ e $(Y, d_Y)$ espaços métricos e $(Y, +Y, \cdot_Y)$ um espaço vetorial sobre $\mathbb{R}$. Se $f : X \to Y$ e $g : X \to Y$ são duas funções $\alpha$-limitadas, então $f + g$ é $(2\alpha)$-limitada e $\lambda f$ é $(\lambda \alpha)$-limitada para todo $\lambda > 0$. \end{theorem} \begin{proof} Se $x \in X$ e $\tilde{x} \in X$, $d_X(x, \tilde{x}) \le 1$, então $f(\tilde{x}) \in \overline{B}_{\alpha}(f(x))$ e $g(\tilde{x}) \in \overline{B}_{\alpha}(g(x))$, portanto $(f + g)(\tilde{x}) = f(\tilde{x}) + g(\tilde{x}) \in \overline{B}_{2\alpha}((f + g)(x))$, e consequentemente $f + g$ é $(2\alpha)$-limitada. Analogamente, $(\lambda f)(\tilde{x}) \in \overline{B}_{\lambda \alpha}(f(x))$ e $\lambda f$ é $(\lambda \alpha)$-limitada. \end{proof}

O teorema acima permite considerar espaços vetoriais de mapas entre dois espaços métricos que são $\alpha$-limitados para algum $\alpha \ge 0$.

O produto de duas funções $\alpha$-limitadas (com o mesmo domínio e contradomínio) pode não ser $\beta$-limitado para nenhum $\beta > 0$, assim como o recíproco de uma função $\alpha$-limitada (quando definido).

\begin{example} $f : \mathbb{N} \to \mathbb{N}$ dada por $f(n) = n$ é $1$-limitada (considerando a métrica usual), mas $f^2$ coincide com $g : \mathbb{R} \to \mathbb{R}$, $g(x) = x^2$, portanto não é $\alpha$-limitada para nenhum $\alpha > 0$. \end{example}

\begin{example} $f : [1, +\infty) \to \mathbb{R}$ dada por $f(x) = \frac{1}{x^2}$ é $1$-limitada, mas $\frac{1}{f}$ coincide com $g : \mathbb{R} \to \mathbb{R}$, $g(x) = x^2$, portanto não é $\alpha$-limitada para nenhum $\alpha > 0$. \end{example}

O teorema de extensão para funções $\alpha$-limitadas não é muito forte.

Os exemplos abaixo ilustram por que as condições do teorema são necessárias.

\begin{example} Seja $X \coloneqq \{(x, \pm 1) \mid x \in \mathbb{Z}\} \subset \mathbb{Z}^2$. Considere $f : X \to \mathbb{Z}$ dada por $f(x, 1) = x$ e $f(x, -1) = 0$ para qualquer $x \in \mathbb{Z}$. Claramente, $f$ é $1$-limitada, considerando a métrica usual, mas $f$ não tem nenhuma extensão $\alpha$-limitada, para qualquer $\alpha$, ao conjunto $\{(x, y) \mid x \in \mathbb{Z}, y \in \{-1,0,1\}\}$, pois $\overline{B}_1((x, 0)) \supset \{ (x, 1), (x, -1) \}$ e $f(x, 1) = x$, $f(x, -1) = 0$. \end{example}

\begin{figure}[H]
  \begin{center}
    \centering
    \begin{tikzpicture}
      % \draw[thick] (-5.8, -1.8) rectangle (9, 1.8);
      \foreach \y in {1, 0, -1} {
          \foreach \x in {-6, ..., 6} {
              \fill[color=black] (\x*1.1+3,\y*1.3) circle (0.06);

              \ifnum\y=1
                \node[right] at (\x*1.1+3,\y*1.3+0.3) {$\x$}; % Label "A" at (0, 0)
              \fi
              \ifnum\y=-1
                \node[right] at (\x*1.1+3,\y*1.3+0.3) {$0$};
              \fi
            }
        }
    \end{tikzpicture}
    \captionsetup{justification=centering}
    \caption{Função $1$-limitada sem extensão $\alpha$-limitada, para qualquer $\alpha > 0$, para um conjunto com infinitos pontos adicionais.}
  \end{center}
\end{figure}


\begin{example} Seja $s_n \in l^2(\mathbb{R})$ a sequência com um $1$ na posição $n$-ésima e $0$'s nas outras posições, e $X = \{ s_n \mid n \in \mathbb{Z}_{> 0} \}$. $f : X \to \mathbb{Z}$ dada por $f(s_n) = n$ é $1$-contínua considerando a norma $l^2$, mas não tem nenhuma extensão $\alpha$-contínua ao conjunto $X \cup \{(0, 0, 0, \dots)\}$ para qualquer $\alpha \ge 0$. \end{example}

\begin{figure}[H]
  \begin{center}
    \begin{tikzpicture}
      \fill (0,0) circle (0.07);
      \node at ([shift={(1cm, -0.5cm)}] 0,0) {$(0, 0, \dots)$};

      % Constants
      \def\n{10} % Number of surrounding points
      \def\k{3.2}  % Constant k in the angle formula
      \def\startOffset{-77}

      % Loop to place points in a circular arrangement
      \foreach \i in {1,...,\n} {
          % Calculate angle for the point using (2 * pi / k) * log(N)
          \pgfmathsetmacro{\angle}{\startOffset + (360 / \k) * ln(\i + 1)}
          \pgfmathsetmacro{\angleout}{\startOffset + (360 / \k) * ln(\i)}

          % Coordinates of the surrounding point at radius 2
          \path (\angle:5) coordinate (P\i);
          \draw[dashed] (0, 0) -- (P\i);
          \ifnum\i>1
            \pgfmathtruncatemacro{\prev}{\i-1}
            % Calculate the start and end angles for each arc
            \pgfmathsetmacro{\startAngle}{\startOffset + (360 / \k) * ln(\prev + 1)}
            \pgfmathsetmacro{\endAngle}{\angle}
            \ifnum\i<4
              \draw[dashed]
              (P\prev) arc[start angle=\startAngle, end angle=\endAngle, radius=5]
              node[midway, above right] {$\sqrt{2}$}; % Label at midpoint of each edge
            \else
              \ifnum\i>4
                \draw[dashed]
                (P\prev) arc[start angle=\startAngle, end angle=\endAngle, radius=5];
              \else
                \draw[dashed]
                (P\prev) arc[start angle=\startAngle, end angle=\endAngle, radius=5]
                node[midway, above] {$\sqrt{2}$}; % Label at midpoint of each edge
              \fi
            \fi
          \fi

          \ifnum\i<5
            \fill[font=\bfseries] (P\i) circle (0.05) node[above right] {$s_{\i}$};
          \else
            \ifnum\i>9
              \fill (P\i) circle (0.05) node[below] {$s_{\i}$};
            \else
              \fill (P\i) circle (0.05) node[above left] {$s_{\i}$};
            \fi
          \fi
        }

      \node at ([shift={(0.5cm, -0.7cm)}] P10) {$\ddots$};

    \end{tikzpicture}
    \captionsetup{justification=centering}
    \caption{Representação informal da sequência de pontos de $l^2(\mathbb{Z}_{> 0})$ na esfera unitária. A escala dos angulos que os pontos $s_n$ formam com o eixo $x$ é logarítimica em $n$.}
  \end{center}
\end{figure}


\begin{theorem} \label{tietze} Sejam $(X,d_X)$ e $(Y, d_Y)$ espaços métricos, $\mathcal{X}, \tilde{\mathcal{X}} \subset X$, onde $\tilde{\mathcal{X}}$ é finito, e seja $f : \mathcal{X} \to Y$ uma função $(\alpha,\lambda)$-limitada. Existe $\beta \ge 0$ tal que $f$ tem uma extensão $(\beta,\lambda)$-limitada para $\mathcal{X} \cup \tilde{\mathcal{X}}$ se e somente se $f$ é limitada em $\overline{B}_{\lambda}(\tilde{x}) \cap \mathcal{X}$ para cada $\tilde{x} \in \tilde{\mathcal{X}}$. \end{theorem}

\begin{proof} A necessidade dessa condição é evidente. Vamos então provar que ela é suficiente.

  Podemos supor que $\tilde{\mathcal{X}} \cap \mathcal{X} = \varnothing$, já que obviamente $f$ é limitada em $\overline{B}_{\lambda x}$ para cada $x \in \mathcal{X}$.

  Procederemos por indução sobre $n$.

  Para $n = 0$, o resultado é óbvio. Suponha que o resultado seja verdade para $n$, e $\tilde{\mathcal{X}}$ tenha $n+1$ elementos.

  Fixe $\tilde{x} \in \tilde{\mathcal{X}}$ e $y \in Y$. Pela hipótese, $f$ tem uma extensão $(\beta,\lambda)$-limitada para $\mathcal{X} \cup (\tilde{\mathcal{X}} \setminus {\tilde{x}})$. Defina $\tilde{f} : \mathcal{X} \cup \tilde{\mathcal{X}}$ por $\tilde{f}(\tilde{x}) = y$ e $\tilde{f}(x) = f(x)$ para cada $x \in \mathcal{X} \cup (\tilde{\mathcal{X}} \setminus {\tilde{x}})$. Claramente existe $\gamma \ge 0$ tal que $\tilde{f}$ é $(\gamma,\lambda)$-limitada em $x$ para cada $x$ diferente de $\tilde{x}$. Mas como $f$ é limitada em $\overline{B}_{\lambda}(\tilde{x}) \cap \mathcal{X}$, existe $M \ge 0$ tal que $\lvert f(x) \rvert \le M$ para cada $x \in \overline{B}_{\lambda}(\tilde{x}) \cap \mathcal{X}$. Portanto, $\lvert \tilde{f}(x)-\tilde{f}(\tilde{x})\rvert \le \lvert f(x)\rvert + \lvert -y\rvert \le M + \lvert-y\rvert$ para cada $x \in \overline{B}_{\lambda}(\tilde{x}) \cap \mathcal{X}$, e consequentemente, $\tilde{f}$ é $(\max\{M + \lvert-y\rvert, \gamma\},\lambda)$-limitada. Assim, o resultado também é verdade para $n+1$. \end{proof}

Este corolário e os exemplos subsequentes ilustram algumas propriedades interessantes das funções $\alpha$-limitadas, especialmente como elas podem ser estendidas de conjuntos compactos para conjuntos finitos.

\begin{corollary} Toda função $\alpha$-limitada definida em um conjunto compacto $\mathcal{X}$ possui uma extensão $\beta$-limitada para qualquer conjunto finito $\tilde{\mathcal{X}} \supseteq \mathcal{X}$. \end{corollary} %%% FIX

\begin{proof} Como $\mathcal{X}$ é compacto, o conjunto $\{ \overline{B}_1(x) \mid x \in \mathcal{X} \} \supseteq \{ B_1(x) \mid x \in \mathcal{X} \} \supseteq \mathcal{X}$ possui uma subcobertura finita, que naturalmente também cobre $\overline{B}_1(x) \cap \mathcal{X} \subset \mathcal{X}$ para todos os $x \in \tilde{\mathcal{X}} \setminus \mathcal{X}$. \end{proof}

Note que podemos ter $\beta < \alpha$ no teorema acima, já que $\alpha$ não precisa ser mínimo e $\tilde{\mathcal{X}} \setminus \mathcal{X}$ pode, por exemplo, ser vazio. Também não podemos assumir que $\beta = \alpha$ no teorema acima.

\begin{example} Seja $X \coloneqq \{-1, 1\} \subset \mathbb{Z}$. A função $f : X \to \mathbb{Z}$ dada por $f(x) = 2x$ é $1$-limitada (considerando a métrica usual), mas não existe uma extensão $1$-limitada de $f$ para o conjunto $\{-1, 0, 1\}$. \end{example}

Um exemplo comum de função $\alpha$-limitada é o seguinte:

\begin{example} Considerando a norma espectral, a função traço $\tr : \mathbb{R}^{m \times n} \to \mathbb{R}$ é $1$-limitada. \end{example}

\newpage

\section{Resultados da teoria de matrizes de Toeplitz} \vspace{1cm}

%% Give a look at l^2(???)

Em \cite{Toeplitz1911} Otto Toeplitz apresenta as $L$-formas, anteriormente estudadas em sua tese de habilitação para a universidade Universidade de Göttingen. Posteriormente seus estudos foram adaptados para o que hoje chamamos de matrizes de Toeplitz.

Uma matriz de Toeplitz é uma matriz possivelmente infinita da forma \[ \begin{pmatrix} a_{0} & a_{-1} & a_{-2} & \cdots \\ a_{1} & a_{0} & a_{-1} & \cdots \\ a_2 & a_{1} & a_{0} & \cdots \\ \vdots & \vdots & \vdots & \ddots \end{pmatrix}\text{.} \]

As matrizes de Toeplitz possuem uma ampla gama de aplicações, que vão desde Teoria dos Operadores, Análise Harmonica, Teoria da Informação e Processamento de Sinais à Mecânica Quântica.

Essas matrizes são especialmente úteis quando a sequência $\{a_n\}$ determina os coeficientes de Fourier de uma determinada função, uma vez que isso possibilita utilizar ferramentas da Análise para calcular, por exemplo, a norma do operador em $l^2(\mathbb{Z}_{> 0})$ que a matriz representa.

Em \cite{Toeplitz1911} Toeplitz mostra uma condição necessária e suficiente para que uma matriz de Toeplitz induza um operador limitado em $l^2(\mathbb{Z}_{> 0})$, é claro, utilizando a linguagem das $L$-formas.

\begin{theorem*}[Operadores Limitados Induzidos por Matrizes de Toeplitz]
  A matriz infinita \[A = \begin{pmatrix} a_{0} & a_{-1} & a_{-2} & \cdots \\ a_{1} & a_{0} & a_{-1} & \cdots \\ a_2 & a_{1} & a_{0} & \cdots \\ \vdots & \vdots & \vdots & \ddots \end{pmatrix}\] induz um operador limitado em $l^2(\mathbb{Z}_{>0})$ se e somente se os números $\{a_n\}$ são os coeficientes de Fourier de uma função integrável $a \in L^\infty(\mathbb{T})$ definida no círculo unitário complexo, \[a_n = \frac{1}{2\pi} \int_{0}^{2\pi} a(e^{i \theta})e^{-i n \theta} \d{\theta}, \quad n \in \mathbb{Z}\text{.}\]
  Nesse caso, a norma do operador $A$ é igual à \[\|a\|_{\infty} \coloneqq \esssup_{t \in \mathbb{T}}\lvert a(t)\rvert\text{.}\] %% Look at notations
\end{theorem*}

Essa formulação e uma prova com notação moderna são encontrados em \cite{bottcher}.

Vamos agora apresentar a definição de uma álgebra $C^\ast$, que generalizam a álgebra de operadores limitados em um espaço de Hilbert. A teoria das álgebras $C^\ast$ é utilizada na prova de diversos resultados não apenas acerca de matrizes de Toeplitz, mas em toda a Teoria dos Operadores.

\begin{definition*} Uma \textit{álgebra $C^\ast$} é uma álgebra de Banach complexa $\mathcal{A}$ munida de uma operação $^\ast : \mathcal{A} \to \mathcal{A}$, chamada de involução, tal que, para todos $a, b \in \mathcal{A}$ e $\lambda \in \mathbb{C}$, as seguintes propriedades são satisfeitas:
  \begin{enumerate} \item $(a + b)^\ast = a^\ast + b^\ast$; \item $(\lambda a)^\ast = \overline{\lambda} a^\ast$; \item $(ab)^\ast = b^\ast a^\ast$; \item $(a^\ast)^\ast = a$; \item $|a^\ast a| = |a|^2$.
  \end{enumerate}
\end{definition*}

Em diversos casos, pode-se utilizar a norma do operador em $l^2(\mathbb{Z}_{> 0})$ induzido por uma matriz de Toeplitz para estudar o limite da sequência das normas de matrizes que são ou estão relacionadas à matrizes de Toeplitz, ou vice-versa. Grande parte dos teoremas a cerca dessa relação são provados usando a teoria das álgebras $C^\ast$, assim como outros resultados da teoria das matrizes de Toeplitz. Essa estratégia é apresentada com profundidade em \cite{bottcher}. Na prova de \ref{th:asymptotic} usamos um resultado básico sobre essa relação.

Isso é particularmente utilizado para resolver sistemas lineares usando o conhecido método da seção finita. Esse método possui diversas aplicações em áreas como o processamento de sinais.

\begin{definition*}
  Uma matriz de Toeplitz $ M \in \mathbb{C}^{n \times n} $ é dita \textit{de banda $ k $} se $ m_{ij} = 0 $ para todo $ i, j \in \{1, \dots, n\} $ tal que $ |i - j| > k $.
\end{definition*}

\begin{theorem*}[Norma de Matrizes de Toeplitz de Banda]
  Seja $ M \in \mathbb{C}^{n \times n} $ uma matriz de Toeplitz de banda $ k $. Então, a norma espectral de $ M $ satisfaz
  \[
    \| M \|_2 \le (2k + 1) \max_{|i - j| \le k} |m_{ij}|.
  \]
\end{theorem*}

\begin{proof}
  Observe que cada linha de $ M $ possui no máximo $ 2k + 1 $ elementos não nulos. Aplicando a desigualdade de Cauchy-Schwarz para cada linha, temos:
  \[
    \| M \|_2 \le \max_{1 \le i \le n} \sum_{j = \max\{1, i - k\}}^{\min\{n, i + k\}} |m_{ij}| \le (2k + 1) \max_{|i - j| \le k} |m_{ij}|.
  \]
\end{proof}

A matriz $ M_{n,k} $ definida em \ref{th:matrix-form-n-to-n} é um exemplo de matriz de Toeplitz de banda $ k $, onde os elementos $ m_{ij} $ são iguais a $ 1 $ se $ |i - j| \le k $ e $ 0 $ caso contrário. A norma espectral dessa matriz satisfaz $ \| M_{n,k} \|_2 = 2k + 1 $, conforme utilizado no \ref{th:asymptotic}.

Desse modo esse teorema pode ser usado para generalizar o limite superior de \ref{th:matrix-form-n-to-n} em casos onde não conseguimos calcular diretamente a norma da matrix de Toeplitz relacionada ao sistema.

As matrizes circulantes são um caso especial de matrizes de Toeplitz, onde cada linha é uma rotação cíclica da linha anterior. Elas possuem propriedades espectrais que facilitam a análise, especialmente no contexto de autovalores e autovetores.

\begin{definition*}
  Uma matriz de Toeplitz $ C \in \mathbb{C}^{n \times n} $ é dita \textit{circulante} se cada linha subsequente é uma rotação cíclica da linha anterior. Ou seja,
  \[
    C = \begin{pmatrix}
      c_0     & c_{n-1} & \cdots  & c_2    & c_1     \\
      c_1     & c_0     & c_{n-1} & \cdots & c_2     \\
      \vdots  & c_1     & c_0     & \ddots & \vdots  \\
      c_{n-2} & \vdots  & \ddots  & \ddots & c_{n-1} \\
      c_{n-1} & c_{n-2} & \cdots  & c_1    & c_0     \\
    \end{pmatrix}.
  \]
\end{definition*}

\begin{theorem*}[Diagonalização de Matrizes Circulantes]
  Toda matriz circulante $ C \in \mathbb{C}^{n \times n} $ pode ser diagonalizada pela matriz de Fourier $ F_n $, ou seja,
  \[
    C = F_n \Lambda F_n^{-1},
  \]
  onde $ \Lambda $ é uma matriz diagonal cujos elementos são os valores de Fourier do vetor de primeira linha de $ C $.

  Em particular, os autovalores de $ C $ são dados por:
  \[
    \lambda_m = \sum_{j=0}^{n-1} c_j \omega_n^{jm}, \quad m = 0, 1, \dots, n-1,
  \]
  onde $ \omega_n = e^{2\pi i / n} $ é a raiz $ n $-ésima da unidade.
\end{theorem*}

A prova desse teorema é sugerida como exercício em \cite[p. 100]{horn2013matrix}.

% \begin{example}
%   Considere a matriz circulante $ C $ de ordem $ 3 $:
%   \[
%     C = \begin{pmatrix}
%       c_0 & c_2 & c_1 \\
%       c_1 & c_0 & c_2 \\
%       c_2 & c_1 & c_0 \\
%     \end{pmatrix}.
%   \]
%   Seus autovalores são:
%   \[
%     \lambda_0 = c_0 + c_1 + c_2, \quad \lambda_1 = c_0 + c_1 \omega_3 + c_2 \omega_3^2, \quad \lambda_2 = c_0 + c_1 \omega_3^2 + c_2 \omega_3,
%   \]
%   onde $ \omega_3 = e^{2\pi i / 3} $.
% \end{example}

As propriedades espectrais simplificadas das matrizes circulantes tornam-nas úteis para entender comportamentos assintóticos em contextos como o do Capítulo 3.

Mostramos agora um resultado importante sobre a inversa de uma matriz de Toeplitz, que é claro, é de grande utilidade na solução dos sistemas lineares mencionados anteriormente.

A fórmula de Gohberg-Semencul fornece uma representação explícita para a inversa de uma matriz de Toeplitz invertível, permitindo calcular inversas aproximadas de matrizes de Toeplitz de grandes dimensões.
\begin{theorem*}[Fórmula de Gohberg-Semencul]
  Seja $ T_n(a)$ uma matriz de Toeplitz $ n \times n$ gerada por um símbolo $ a \in L^\infty(\mathbb{T})$, tal que $ T_n(a)$ é invertível. Então, a inversa de $ T_n(a)$ pode ser expressa como
  \[
    T_n(a)^{-1} = X_n - U_n V_n^T,
  \]
  onde $ X_n$ é uma matriz de Toeplitz gerada pelo símbolo $1 / a$, e $ U_n$, $ V_n$ são vetores apropriados em $ \mathbb{C}^n$.
\end{theorem*}
Para uma prova, eja \cite{gohberg1972}.
Essa fórmula é particularmente útil em problemas de processamento de sinais e resolução de sistemas lineares com estruturas especiais.

Um resultado importante provado por Szegö em \cite{szego1915} e futuramente melhorado por ele e diversos outros matemáticos é o seguinte:
\begin{theorem*}[Teorema de Limite de Szegő]
  Seja $A$ a matriz de Toeplitz correspondente à função real $a \in L^1(\mathbb{T})$, e $A_n$ sua seção finita $n \times n$. Se $\log a \in L^1(\mathbb{T})$ e $a(x) > 0$ para todo $x \in \mathbb{T}$, então \[\lim_{n\to \infty} \frac{\det A_n}{\det A_{n-1}} = \exp\left(\frac{1}{2\pi} \int_{0}^{2\pi} \log a\left(e^{i\theta}\right) \right)\text{.}\]
\end{theorem*}

Esse teorema e principalmente suas versões mais fortes tem aplicações na caracterização assintótica do espectro das matrizes $A_n$.

Como mencionado, sob certas condições as matrizes de Toeplitz definem um operador limitado em $l^2(\mathbb{Z}_{\ge 0})$. Evidentemente generalizações desse fato foram estudadas para espaços de Banach. Isso motiva a definição de operadores de Toeplitz.

Um operador de Toeplitz é um operador linear contínuo definido em um espaço de Hilbert, frequentemente relacionado a funções analíticas no círculo unitário. A conexão entre matrizes de Toeplitz e operadores de Toeplitz é estabelecida através da identificação de matrizes finitas com operadores lineares em espaços de dimensão finita.
\begin{definition*}
  Seja $H^2(\mathbb{T})$ o espaço de Hardy de funções na circunferência unitária $\mathbb{T}$. Para uma função limitada $a \in L^\infty(\mathbb{T})$, o \textit{operador de Toeplitz} $T(a) : H^2(\mathbb{T}) \to H^2(\mathbb{T})$ é definido por $T(a)f=P(af)$, onde $P : L^2(\mathbb{T}) \to H^2(\mathbb{T})$ é o operador de projeção de Szegő.
\end{definition*}
O símbolo $a$ do operador de Toeplitz $T(a)$ é a função em $L^\infty(\mathbb{T})$ a partir da qual o operador é definido. Esse símbolo desempenha um papel crucial na determinação das propriedades do operador.

Apresentamos agora um resultado importante sobre o espectro de operadores de Toeplitz. O teorema do mapeamento espectral relaciona o espectro de um operador com o espectro de sua imagem através de uma função contínua.
\begin{theorem*}[Teorema do Espectro de Toeplitz]
  Seja $T(a)$ um operador de Toeplitz com símbolo $a \in L^\infty(\mathbb{T})$. Então, o espectro de $T(a)$, denotado por $\sigma(T(a))$, satisfaz $\sigma(T(a))=a(T)$, onde $\overline{a(\mathbb{T})}$ é o fecho da imagem de $a$ em $\mathbb{C}$.
\end{theorem*}

Veja \cite[p. 85]{Douglas1998} para uma prova desse teorema.

Esse resultado permite determinar o espectro de um operador de Toeplitz diretamente a partir de seu símbolo, facilitando a análise espectral desses operadores.

Os operadores de Toeplitz possuem aplicações em problemas do valor de borda de funções analíticas e equações integrais singulares.

Por fim, apresentados a fatoração de Wiener-Hopf, uma técnica que permite decompor uma função em fatores que são analíticos dentro e fora do círculo unitário, o que é essencial na solução de certas equações integrais e na teoria de operadores de Toeplitz.
\begin{theorem*}[Fatoração de Wiener-Hopf]
  Se $a \in L^\infty(\mathbb{T})$ é invertível e $\log a \in L^1(\mathbb{T})$, então existe uma fatoração única $a(e^{i\theta}) = a_-(e^{i\theta}) a_+(e^{i\theta})$, onde $a_-$ e $a_+$ são funções com extensões analíticas dentro e fora do círculo unitário, respectivamente.
\end{theorem*}
Para uma prova, veja \cite{wiener1931}.

A fatoração de Wiener-Hopf é aplicada na resolução de problemas em física matemática e engenharia, onde a estrutura de convolução é predominante.

Os resultados apresentados neste capítulo destacam a profundidade e a riqueza da teoria de matrizes de Toeplitz. Como dito esses objetos possuem uma ampla gama de aplicações em diversas áreas, e são de extrema importância para a análise assintótica de diversas quantidades.


\newpage

\section{Contagem de funções \texorpdfstring{$k$}{k}-limitadas em \texorpdfstring{$[n]$}{[n]}} \vspace{1cm} Muitos problemas em combinatória podem ser descritos como um sistema de recorrências lineares homogêneas de primeira ordem, como mostrado em \cite{coulson}. Todo sistema de recorrências lineares homogêneas de primeira ordem pode ser expresso na forma de matriz. Às vezes, também é útil considerar sequências desses sistemas em $n$ recorrências, bem como entender seus comportamentos assintóticos à medida que $n \to \infty$, mas isso geralmente é uma tarefa difícil. Os resultados deste capítulo são úteis para entender o comportamento assintótico de sistemas cuja matriz correspondente é uma matriz de Toeplitz.

Para cada par de inteiros positivos $n$ e $k$, $a(n,k)$ denota a quantidade de funções $k$-limitadas de $[n]$ para $[n]$.

Nós utilizaremos a notação $\chi_k(i, j) = \begin{cases} 1 & \text{se } \lvert i - j\rvert \le k \\ 0 & \text{se } \lvert i - j\rvert > k \end{cases}$.

Também definiremos $M_{n,k} \in \mathbb{Z}^{n \times n}$ para denotar a matriz definida por $m_{ij} = \chi_k(i, j)$.

\begin{theorem} \label{th:matrix-form-n-to-n} $a(n,k)$, para $0 \le k \le n - 1$, é dado por $\mathbf{1}_n^\intercal M_{n,k}^{n-1} \mathbf{1}_n$. \end{theorem}

\begin{proof}

  Provaremos o fato mais forte de que o número de funções $k$-limitadas $f : [r] \to [n]$, $r \in [n]$, é dado por $\mathbf{1}_n^\intercal M_{n,k}^{r - 1} \mathbf{1}_n$, e que o número de tais funções com $f(r) = i$ é dado por $(M_{n,k}^{r - 1} \mathbf{1}_n)_i$. Procederemos por indução sobre $r$.

  %% IMAGE PLEASE

  %% Send what I'll do to Martineli
  De fato, se $r = 1$, toda função $f : [r] \to [n]$ é constante, portanto há $n$ tais funções (uma para cada elemento de $[n]$), e $n = \mathbf{1}_n^\intercal I \mathbf{1}_n = \mathbf{1}_n^\intercal M_{n,k}^{1 - 1} \mathbf{1}_n$, como queríamos.

  Se isso é válido para $r - 1$, então podemos encontrar o número de funções $f : [r] \to [n]$ contando o número de funções $k$-limitadas $g : [r-1] \to [n]$ das quais elas podem ser extensões. Se $f(r) = i$, então $f(r - 1) \in \{i - k, i - k + 1, \dots, i + k\} \cap [n]$, portanto a quantidade de tais funções $f$ é dada pela soma da quantidade de funções $k$-limitadas $g : [r-1] \to [n]$ com $g(r-1) \in \{i - k, i - k + 1, \dots, i + k\} \cap [n]$, que é $\sum_{j = 0}^{n-1} \chi_k(i, j) (M_{n,k}^{r-1} \mathbf{1}_n)_i$, que pela definição de multiplicação de matrizes é igual a $(M_{n,k} (M_{n,k}^{r-1} \mathbf{1}_n))_i = (M_{n,k}^r \mathbf{1}_n)_i$. Consequentemente, o número de todas as funções $k$-limitadas $f : [r] \to [n]$ é dado por $\mathbf{1}_n^\intercal M_{n,k}^r \mathbf{1}_n$, e o resultado segue para $n$, como queríamos.

\end{proof}

Abaixo estão alguns valores usando esta fórmula.

\begin{table}[H]
  \centering
  \adjustbox{width=\textwidth}{
    \begin{tabular}{cccccccccccc}
      \toprule
      \diagbox{$n$}{$k$} & $0$  & $1$       & $2$         & $3$          & $4$           & $5$            & $6$            & $7$             & $8$             & $9$             & $10$            \\
      \midrule
      $2$                & $2$  & $4$       &             &              &               &                &                &                 &                 &                 &                 \\
      $3$                & $3$  & $17$      & $27$        &              &               &                &                &                 &                 &                 &                 \\
      $4$                & $4$  & $68$      & $178$       & $256$        &               &                &                &                 &                 &                 &                 \\
      $5$                & $5$  & $259$     & $1161$      & $2309$       & $3125$        &                &                &                 &                 &                 &                 \\
      $6$                & $6$  & $950$     & $7310$      & $20650$      & $35954$       & $46656$        &                &                 &                 &                 &                 \\
      $7$                & $7$  & $3387$    & $44787$     & $182349$     & $408623$      & $654797$       & $823543$       &                 &                 &                 &                 \\
      $8$                & $8$  & $11814$   & $267802$    & $1569700$    & $4609826$     & $9041498$      & $13667858$     & $16777216$      &                 &                 &                 \\
      $9$                & $9$  & $40503$   & $1569883$   & $13249619$   & $51374875$    & $123984439$    & $222457927$    & $321839625$     & $387420489$     &                 &                 \\
      $10$               & $10$ & $136946$  & $9051480$   & $109906170$  & $561224422$   & $1687725824$   & $3592094120$   & $6040291396$    & $8441614754$    & $100000000000$  &                 \\
      $11$               & $11$ & $5035745$ & $566193771$ & $9871777663$ & $66390984341$ & $249898450197$ & $634363172685$ & $1235414492037$ & $1976196614909$ & $2685194913379$ & $3138428376721$ \\
      % \bottomrule
    \end{tabular}
  }
  \caption{
  \pt{Número de funções $f \colon [n] \to [n]$ $k$-limitadas}
  \en{Number of $k$-bounded functions $f \colon [n] \to [n]$}
  }
  \label{data2}
\end{table}


O código para calcular esses valores pode ser encontrado em \cite{github}. Valores muito maiores podem ser computados. Por exemplo, $a(200,199)$ (que resulta em um número com $463$ dígitos) pode ser computado em menos de $15$ segundos em um computador moderno.

Agora apresentamos a caracterização assintótica desta fórmula.

\begin{theorem} \label{th:asymptotic} Seja $k$ um inteiro positivo fixo. Então, $a(n, k) \sim n(2k+1)^{n-1}$. \end{theorem} \begin{proof} Claramente, $M_{n,k}$ é a matriz de adjacência de um grafo com $n$ vértices, com o grau médio $\overline{d}$ igual a $(2k + 1) - \mathcal{O}(\frac{1}{n})$. Agora, pela desigualdade de Ërdos-Simonovits (veja \cite{erdos-simonovits}), $n \overline{d}^{n-1} \le w_{n-1}$, onde $w_{n-1}$ é a quantidade de caminhos com comprimento $n-1$ no grafo. É fácil ver que $w_{n-1} = a(n, k)$. Portanto, $n ((2k + 1) - \mathcal{O}(\frac{1}{n}))^{n-1} \le a(n, k)$.

  Seja \[M_k = \begin{pmatrix} a_{0} & a_{-1} & a_{-2} & \cdots \\ a_{1} & a_{0} & a_{-1} & \cdots \\ a_2 & a_{1} & a_{0} & \cdots \\ \vdots & \vdots & \vdots & \ddots \end{pmatrix}\] a matriz infinita definida por $m_{ij} = \chi_k(i, j)$.

  Claramente apenas uma quantidade finita de termos de $\{a_n\}$ são não nulos, e consequentemente os termos de $\{ a_n \}$ são os coeficientes de Fourier do polinômio trigonométrico $a(x) = \sum_{n = -N}^N a_n e^{i n x}$, e pelo resultado bem conhecido de Toeplitz \cite{Toeplitz1911}, $M_k$ define um operador limitado em $l^2(\mathbb{Z}_{\ge 0})$, então $\|M_k\|_2 = \| a \|_{\infty}$ existe, já que todo polinômio trigonométrico é contínuo e consequentemente limitado em $\mathbb{T}$.

  Agora, pela desigualdade de Cauchy e a propriedade submultiplicativa de $\| \cdot \|_2$, temos que
  \begin{align*}
    \mathbf{1}_n^\intercal M_{n,k}^{n-1} \mathbf{1}_n & = \lvert \mathbf{1}_n^\intercal M_{n,k}^{n-1} \mathbf{1}_n\rvert      \\
                                                      & \le \| \mathbf{1}_n^\intercal \|_2 \| M_{n,k}^{n-1} \mathbf{1}_n \|_2 \\
                                                      & = \sqrt{n} \| M_{n,k}^{n-1} \mathbf{1}_n \|_2                         \\
                                                      & \le \sqrt{n} \| M_{n,k}^{n-1} \|_2 \| \mathbf{1}_n \|_2               \\
                                                      & = n \| M_{n,k}^{n-1} \|_2                                             \\
                                                      & \le n \| M_{n,k} \|_2^{n-1}
  \end{align*}

  Pelo resultado provado em, por exemplo, \cite[p. 52]{bottcher}, temos que $\| M_{n,k} \| \le \| M_k \|$.

  Mas os coeficientes de Fourier de $a$ são os coeficientes de Fourier do kernel de Dirichlet $D_k$, e $\| a \|_{\infty} = \esssup{t \in T} \lvert a(t)\rvert = \sup_{t \in T} \lvert a(t)\rvert$. É fato conhecido que $\sup_{t \in T} \lvert D_k(t)\rvert = 2k + 1$. Portanto, $\| M_k \| = 2k + 1$.

  Concluímos a prova aplicando o teorema do aperto a ambas as desigualdades. \end{proof}

Observe que o limite inferior da prova acima pode ser facilmente estendido a qualquer matriz que seja simétrica e tenha entradas não negativas usando a desigualdade de Blakley-Roy (veja \cite{blakley-roy}).

Note que o limite inferior poderia também ser estabelecido usando somente resultados referentes à matrizes de Toeplitz, mas optamos por apresentar a prova aqui presente devido a maior facilidade da sua adaptação a uma gama maior de matrizes.

\newpage
\section*{Próximos Passos}
\addcontentsline{toc}{section}{Próximos Passos}
\vspace{1cm}

Pode-se usar o teorema de Blakley-Roy para provar uma versão mais forte do limite inferior do \ref*{th:asymptotic} aplicado a uma classe mais ampla de matrizes de Toeplitz que não requer que os coeficientes da matriz assumam valores apenas em $\{0, 1\}$. Isso pode ser combinado com uma análise dos casos de igualdade da desigualdade de Bernstein, que afirma que se $a$ é um polinômio trigonométrico com coeficientes de Fourier $\{a_n\}$, então $\|a\|_\infty \le \sum_{n \in \mathbb{Z}} \lvert a_n\rvert$, para provar uma versão mais forte da caracterização assintótica, novamente aplicada a matrizes de Toeplitz mais gerais.

Também é possível estudar as propriedades topológicas das funções $(\alpha, \lambda)$-limitadas, possivelmente entendendo se é o caso que o espaço topológico induzido pelo espaço vetorial das aplicações $(\alpha, \lambda)$-limitadas entre dois espaços métricos herda propriedades dos espaços topológicos induzidos por esses espaços métricos.

É evidente que também é desejável a caracterização assintótica da quantidade de funções $k$-limitadas de $[n]$ para $[m]$. Acreditamos que o uso de matrizes de Toeplitz de blocos podem ser úteis para obter esse resultado.

Apesar de não ser utilizada neste trabalho, acreditamos que a Teoria Espectral dos Grafos pode ser usada em conjunto com matrizes de Toeplitz para provar uma ampla gama de resultados a cerca da caracterização assintótica de soluções de sistemas de recorrências, uma vez que, ao fornecer informações sobre os autovalores das matrizes associadas à esses sistemas, permite uma abordagem mais direta das aproximações envolvidas. Em particular acreditamos que essa teoria é essencial ao abordar resultados mais gerais sobre por exemplo a quantidade de funções $k$-limitadas de $[n]^n$ para $[n]$ e para $[n]^n$.

\newpage
\normalsize
\bibliography{refs}

\end{document}
