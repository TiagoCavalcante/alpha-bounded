\begin{center}
  \large \textbf{Resumo}
\end{center}
\vspace{1cm}
\par Neste trabalho, introduzimos o conceito de funções $(\alpha, \lambda)$-limitadas, caracterizadas por uma variação local limitada, especialmente úteis em conjuntos discretos. Inicialmente, definimos formalmente essas funções e investigamos suas propriedades fundamentais, destacando diferenças significativas em relação às funções contínuas. O principal resultado obtido é a caracterização assintótica de $a(n, k)$, que representa o número de funções de $[n]$ para $[n]$ que são $k$-limitadas considerando a distância de Manhattan. A prova desse resultado combina matrizes de Toeplitz com uma desigualdade bem conhecida da teoria dos grafos. Além disso, enfatizamos a relevância da interseção entre combinatória algébrica, teoria dos grafos e análise de espaços métricos discretos, sugerindo pesquisas futuras sobre as propriedades topológicas dessas funções e suas potenciais aplicações em contextos computacionais e matemáticos.

\vspace{1cm}

\begin{raggedleft}
  \pt{\textbf{Palavras-chave}: espaços métricos; matrizes de Toeplitz; estimativas assintóticas; teoria dos grafos; combinatória algébrica.}
  \en{\textbf{Palavras-chave}: espaços métricos; matrizes de Toeplitz; teoria dos grafos; combinatória algébrica; estimativas assintóticas.}
\end{raggedleft}
