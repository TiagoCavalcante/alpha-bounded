\section*{Notações}
% \addcontentsline{toc}{section}{Notações}

\begin{description}[leftmargin=!, labelwidth=\widthof{\texttt{Função $f \colon X \to Y$}}]
  \pt{\item[{$[n]$}] Conjunto $\{1, 2, \dots, n\}$.}
  \en{\item[{$[n]$}] Set $\{1, 2, \dots, n\}$.}

  \pt{\item[$\mathbb{Z}_{>0}$] Conjunto dos inteiros positivos.}
  \en{\item[$\mathbb{Z}_{>0}$] Set of positive integers.}

  \pt{\item[$(X, d_X)$] Espaço métrico $X$ com a métrica $d_X$.}
  \en{\item[$(X, d_X)$] Metric space $X$ with metric $d_X$.}

  \pt{\item[$\mathbf{1}_n$] Vetor coluna de dimensão $n$ com todos os elementos iguais a $1$.}
  \en{\item[$\mathbf{1}_n$] Column vector of dimension $n$ with all elements equal to $1$.}

  \pt{\item[$(v)_i$] O $i$-ésimo elemento de um vetor $v$.}
  \en{\item[$(v)_i$] The $i$-th element of a vector $v$.}

  \pt{\item[$\| \cdot \|_2$] Norma espectral de uma matriz ou vetor.}
  \en{\item[$\| \cdot \|_2$] Spectral norm of a matrix or vector.}

  \pt{\item[$H^2(\mathbb{T})$] Espaço de Hardy das funções analíticas na circunferência unitária $\mathbb{T}$.}
  \en{\item[$H^2(\mathbb{T})$] Hardy space of analytic functions on the unit circle $\mathbb{T}$.}

  \pt{\item[$T(a)$] Operador de Toeplitz com símbolo $a \in L^\infty(\mathbb{T})$, definido por $T(a)f = P(af)$ onde $P$ é a projeção de Szegő.}
  \en{\item[$T(a)$] Toeplitz operator with symbol $a \in L^\infty(\mathbb{T})$, defined by $T(a)f = P(af)$ where $P$ is the Szegő projection.}

  \pt{\item[$\sigma(T(a))$] Espectro do operador de Toeplitz $T(a)$.}
  \en{\item[$\sigma(T(a))$] Spectrum of the Toeplitz operator $T(a)$.}

  \pt{\item[$\{a_n\}$] Sequência indexada por $n \in \mathbb{Z}$, ou seja, uma sequência de termos $\{a_n : n \in \mathbb{Z}\}$ onde $n$ varia em todos os inteiros.}
  \en{\item[$\{a_n\}$] Sequence indexed by $n \in \mathbb{Z}$, that is, a sequence of terms $\{a_n : n \in \mathbb{Z}\}$ where $n$ ranges over all integers.}

  \pt{\item[$\Lambda$] Matriz diagonal resultante da diagonalização de uma matriz circulante através da matriz de Fourier.}
  \en{\item[$\Lambda$] Diagonal matrix resulting from the diagonalization of a circulant matrix via the Fourier matrix.}

  \pt{\item[$D_k(t)$] Kernel de Dirichlet de ordem $k$.}
  \en{\item[$D_k(t)$] Dirichlet kernel of order $k$.}

  \pt{\item[$T_n(a)$] Matriz de Toeplitz $n \times n$ gerada pelo símbolo $a \in L^\infty(\mathbb{T})$.}
  \en{\item[$T_n(a)$] $n \times n$ Toeplitz matrix generated by symbol $a \in L^\infty(\mathbb{T})$.}

  \pt{\item[$l^2(\mathbb{Z}_{> 0})$] Espaço de sequências quadrado-somáveis indexadas por $\mathbb{Z}_{> 0}$.}
  \en{\item[$l^2(\mathbb{Z}_{> 0})$] Space of square-summable sequences indexed by $\mathbb{Z}_{> 0}$.}

  \pt{\item[$a(x)$] Função símbolo associada a uma matriz de Toeplitz, geralmente definida como um polinômio trigonométrico.}
  \en{\item[$a(x)$] Symbol function associated with a Toeplitz matrix, usually defined as a trigonometric polynomial.}

  \pt{\item[$L^\infty(\mathbb{T})$] Espaço das funções definidas na circunferência unitária complexa $\mathbb{T}$ essencialmente limitadas.}
  \en{\item[$L^\infty(\mathbb{T})$] Space of essentially bounded functions defined on the complex unit circle $\mathbb{T}$.}

  \pt{\item[$\esssup$] Supremo essencial.}
  \en{\item[$\esssup$] Essential supremum.}

  \pt{\item[$\overline{B}_{\lambda}(x)$] Bola fechada de raio $\lambda$ centrada em $x$ no espaço métrico considerado.}
  \en{\item[$\overline{B}_{\lambda}(x)$] Closed ball of radius $\lambda$ centered at $x$ in the considered metric space.}

  \pt{\item[$a(n) \sim b(n)$] $a(n)$ e $b(n)$ são assintoticamente equivalentes, isto é, $\lim_{n \to \infty} \frac{a(n)}{b(n)} = 1$.}
  \en{\item[$a(n) \sim b(n)$] $a(n)$ and $b(n)$ are asymptotically equivalent, that is, $\lim_{n \to \infty} \frac{a(n)}{b(n)} = 1$.}

  \pt{\item[$f \in L^1(\mathbb{T})$] Função $f$ integrável definida na circunferência unitária $\mathbb{T}$.}
  \en{\item[$f \in L^1(\mathbb{T})$] Integrable function $f$ defined on the complex unit circle $\mathbb{T}$.}

  \pt{\item[$\mathcal{O}(\cdot)$] Notação Big-O, indicando que uma função $f(n)$ cresce no máximo na ordem de outra função $g(n)$, ou seja, $f(n) = \mathcal{O}(g(n))$ se existem constantes $C, n_0 > 0$ tais que $|f(n)| \leq C \cdot |g(n)|$ para todo $n \geq n_0$.}
  \en{\item[$\mathcal{O}(\cdot)$] Big-O notation, indicating that a function $f(n)$ grows at most on the order of another function $g(n)$, that is, $f(n) = \mathcal{O}(g(n))$ if there exist constants $C, n_0 > 0$ such that $|f(n)| \leq C \cdot |g(n)|$ for all $n \geq n_0$.}
\end{description}
