\section*{Notações}
% \addcontentsline{toc}{section}{Notações}

\begin{description}[leftmargin=!, labelwidth=\widthof{\texttt{Função $f : X \to Y$}}]
  \item[{$[n]$}] Conjunto $\{1, 2, \dots, n\}$.

  \item[$\mathbb{Z}_{>0}$] Conjunto dos inteiros positivos.

  \item[$(X, d_X)$] Espaço métrico $X$ com a métrica $d_X$.

  \item[$\mathbf{1}_n$] Vetor coluna de dimensão $n$ com todos os elementos iguais a $1$.

  \item[$\| \cdot \|_2$] Norma espectral de uma matriz ou vetor.

  \item[$H^2(\mathbb{T})$] Espaço de Hardy das funções analíticas na circunferência unitária $\mathbb{T}$.

  \item[$T(a)$] Operador de Toeplitz com símbolo $a \in L^\infty(\mathbb{T})$, definido por $T(a)f = P(af)$ onde $P$ é a projeção de Szegő.

  \item[$\sigma(T(a))$] Espectro do operador de Toeplitz $T(a)$.

  \item[$\Lambda$] Matriz diagonal resultante da diagonalização de uma matriz circulante através da matriz de Fourier.

  \item[$\omega_n$] Raiz $n$-ésima da unidade, $\omega_n = e^{2\pi i / n}$.

  \item[$D_k(t)$] Kernel de Dirichlet de ordem $k$.

  \item[$T_n(a)$] Matriz de Toeplitz $n \times n$ gerada pelo símbolo $a \in L^\infty(\mathbb{T})$.

  \item[$l^2(\mathbb{Z}_{> 0})$] Espaço de sequências quadrado-somáveis indexadas por $\mathbb{Z}_{> 0}$.

  \item[$a(x)$] Função símbolo associada a uma matriz de Toeplitz, geralmente definida como um polinômio trigonométrico.

  \item[$L^\infty(\mathbb{T})$] Espaço das funções essencialmente limitadas na circunferência unitária $\mathbb{T}$.

  \item[$\esssup$] Supremo essencial.

  \item[$\overline{B}_{\lambda}(x)$] Bola fechada de raio $\lambda$ centrada em $x$ no espaço métrico considerado.

  \item[$a(n) \sim b(n)$] $a(n)$ e $b(n)$ são assintoticamente equivalentes, isto é, $\lim_{n \to \infty} \frac{a(n)}{b(n)} = 1$.

  \item[$f \in L^1(\mathbb{T})$] Função $f$ integrável definida na circunferência unitária $\mathbb{T}$.

  \item[$\overline{a(\mathbb{T})}$] Fecho da imagem da função $a$ no conjunto dos números complexos $\mathbb{C}$.

  \item[$\mathcal{O}(\cdot)$] Notação Big-O, indicando que uma função $f(n)$ cresce no máximo na ordem de outra função $g(n)$, ou seja, $f(n) = \mathcal{O}(g(n))$ se existem constantes $C, n_0 > 0$ tais que $|f(n)| \leq C \cdot |g(n)|$ para todo $n \geq n_0$.
\end{description}

\newpage
